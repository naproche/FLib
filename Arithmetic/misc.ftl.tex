\begin{forthel}
    \begin{proposition}\printlabel{ARITHMETIC_11_8426075493236736}
      Let $n, m, k$ be natural numbers such that $n, m \neq 0$ and $k > 1$.
      Then \[ k^{n} \mid k^{m} \iff n \leq m. \]
    \end{proposition}
    \begin{proof}
      Case $k^{n} \mid k^{m}$.
        Assume $n > m$.
        Take a nonzero natural number $l$ such that $n = m + l$.
        Then $k^{n}
          = k^{m + l}
          = k^{m} \cdot k^{l}$.
        Hence $k^{m} \mid k^{n}$.
        Thus $k^{m} = k^{n}$.
        Therefore $m = n$.
        Contradiction.
      End.

      Case $n \leq m$.
        Take a natural number $l$ such that $m = n + l$.
        Then $k^{m}
          = k^{n + l}
          = k^{n} \cdot k^{l}$.
        Hence $k^{n} \mid k^{m}$.
      End.
    \end{proof}
  \end{forthel}

  \begin{forthel}
    \begin{proposition}\printlabel{ARITHMETIC_11_797196163219456}
      Let $n$ be a composite natural number.
      Then $n$ has a nontrivial divisor $m$ such that $m^{2} \leq n$.
    \end{proposition}
    \begin{proof}
      Define $\Phi = \{ m \in \Nat \mid m$ is a nontrivial divisor of $n \}$.
      $A$ contains some natural number.
      Hence we can take a least natural number $m$ of $A$.
      Consider a natural number $k$ such that $m \cdot k = n$.
      Then $m \leq k$.
      Indeed if $k < m$ then $k$ is the least natural number of $A$.
      Hence $m^{2} = m \cdot m \leq m \cdot k = n$.
      Therefore $m^{2} \leq n$.
    \end{proof}
  \end{forthel}

  \begin{forthel}
    \begin{proposition}\printlabel{ARITHMETIC_11_2632282722533376}
      Let $k$ be a nonzero natural number.
      If $k \cdot m^{2} = n^{2}$ for some nonzero $n, m \in \Nat$ then $k$ is
      compound.
    \end{proposition}
    \begin{proof}
      Assume $k \cdot m^{2} = n^{2}$ for some nonzero $n, m \in \Nat$.
      Consider nonzero $n, m \in \Nat$ such that $k \cdot m^{2} \cdot m^{2} =
      n^{n}$.

      Case $k = 1$. Obvious.

      Case $k > 1$.
        Define $\Phi = \{ n' \in \Nat \mid$ if $n' \neq 0$ and $k \cdot m^{2} =
        n'^{2}$ then $k$ is compound $\}$.

        Let us show that for all $n' \in \Nat$ if $\Phi$ contains all
        predecessors of $n'$ then $\Phi$ contains $n'$.
          Let $n'$ be a natural number.
          Presume that $\Phi$ contains all predecessors of $n'$.
          Assume $n' \neq 0$ and $k \cdot m^{2} = n'^{2}$.

          Suppose that $k$ is prime.
          $k$ is a nontrivial divisor of $n'^{2}$.
          Hence $k$ divides $n'$.
          Take a natural number $l$ such that $k \cdot l = n'$.

          (1) Then $m^{2} = k \cdot l^{2}$.
          Indeed $k \cdot m^{2}
            = (k \cdot l)^{2}
            = k \cdot (k \cdot l^{2})$.

          (2) $m$ is an element of $\Phi$. \\
          Proof.
            We have $n'^{2} > m^{2}$.
            Indeed $k \cdot m^{2} = n'^{2}$ and $k > 1$ and $m^{2} > 0$.
            Hence $m < n'$.
            Indeed if $n' \leq m$ then $n'^{2} \leq m^{2}$.
            Thus $m \in \Phi$.
          Qed.

          (3) $l$ is nonzero.
          Indeed $l = 0 \implies
          m^{2}
            = k \cdot 0^{2}
            = k \cdot 0
            = 0$
          and $m^{2} = 0 \implies m = 0$.

          Therefore $k$ is compound (by 1, 2, 3).
          Contradiction.
        End.

        Thus $\Phi$ contains every natural number.
        Therefore $k$ is compound.
      End.
    \end{proof}
  \end{forthel}
