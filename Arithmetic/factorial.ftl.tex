\documentclass[../arithmetic.tex]{subfiles}

\begin{document}
  \chapter{Factorial}\label{chapter:factorial}

  \filename{arithmetic/sections/11_factorial.ftl.tex}

  \begin{forthel}
    %[prove off][check off]

    [readtex \path{arithmetic/sections/19_exponentiation.ftl.tex}]

    %[prove on][check on]
  \end{forthel}


  \section{Definition of factorial}

  \begin{forthel}
    \begin{lemma}\printlabel{ARITHMETIC_11_353784742019072}
      There exists a map $\varphi$ from $\Nat$ to $\Nat$ such that
      $\varphi(0) = 1$ and $\varphi(n + 1) = \varphi(n) \cdot (n + 1)$ for all
      $n \in \Nat$.
    \end{lemma}
    \begin{proof}
      For all objects $n, m$ we have $(n,m) \in \Nat \times \Nat$ iff $n, m \in
      \Nat$.
      Define $f(n,m) = m \cdot (n + 1)$ for $(n,m) \in \Nat \times \Nat$.
      Take $A = \Nat \times \Nat$ and $a = (0,1)$.
      Then $A$ is a set and $a \in A$.
      Define $g(n,m) = (n + 1, f(n,m))$ for $(n,m) \in \Nat \times \Nat$.
      $g : A \to A$.
      Indeed $g(n,m) \in A$ for every $n, m \in \Nat$.
      Consider a $\psi : \Nat \to A$ such that $\psi$ is recursively defined by
      $a$ and $g$.
      Define $\varphi(n) = \pi_{2}(\psi(n))$ for $n \in \Nat$.
      Then $\varphi$ is a map from $\Nat$ to $\Nat$.

      (1) $\varphi(0) = 1$.
      Indeed $\varphi(0)
        = \pi_{2}(\psi(0))
        = \pi_{2}(a)
        = 1$.

      Let us show that $\pi_{1}(\psi(n)) = n$ for all $n \in \Nat$.
        Define $\Phi = \{ n \in \Nat \mid \pi_{1}(\psi(n)) = n \}$.

        (a) $0 \in \Phi$.
        Indeed $\pi_{1}(\psi(0))
          = \pi_{1}(a)
          = 0$.

        (b) For all $n \in \Phi$ we have $n + 1 \in \Phi$. \\
        Proof.
          Let $n \in \Phi$.
          Then $\pi_{1}(\psi(n)) = n$.
          Hence $\pi_{1}(\psi(n + 1))
            = \pi_{1}(g(\psi(n)))
            = \pi_{1}(\psi(n)) + 1
            = n + 1$.
        Qed.
      End.

      Let us show that $f(\psi(n)) = f(n, \varphi(n))$ for all $n \in \Nat$.
        Define $\Psi = \{ n \in \Nat \mid f(\psi(n)) = f(n, \varphi(n)) \}$.

        (c) $0 \in \Psi$.
        Indeed $f(\psi(0))
          = f(0,1)
          = 1 \cdot (0 + 1)
          = 1
          = \varphi(0)
          = \varphi(0) \cdot (0 + 1)
          = f(0, \varphi(0))$.

        (d) For all $n \in \Psi$ we have $n + 1 \in \Psi$. \\
        Proof.
          Let $n \in \Psi$.
          Then $f(\psi(n)) = f(n, \varphi(n))$.
          Hence $f(\psi(n + 1))
            = f(g(\psi(n)))
            = f(\pi_{1}(\psi(n)) + 1, f(\psi(n)))
            = f(n + 1, f(\psi(n)))
            = f(n + 1, \pi_{2}(g(\psi(n))))
            = f(n + 1, \pi_{2}(\psi(n + 1)))
            = f(n + 1, \varphi(n + 1))$.
          Thus $n + 1 \in \Psi$.
        Qed.
      End.

      (2) For all $n \in \Nat$ we have $\varphi(n + 1) =
      \varphi(n) \cdot (n + 1)$. \\
      Proof.
        Let $n \in \Nat$.
        Then $\varphi(n + 1)
          = \pi_{2}(\psi(n + 1))
          = \pi_{2}(g(\psi(n)))
          = \pi_{2}(\pi_{1}(\psi(n)) + 1, f(\psi(n)))
          = f(\psi(n))
          = f(n, \varphi(n))
          = \varphi(n) \cdot (n + 1)$.
      Qed.
    \end{proof}
  \end{forthel}

  \begin{forthel}
    \begin{lemma}\printlabel{ARITHMETIC_11_5176070548488192}
      Let $\varphi, \varphi' : \Nat \to \Nat$.
      Assume $\varphi(0) = 1$ and $\varphi(n + 1) = \varphi(n) \cdot (n + 1)$
      for all $n \in \Nat$.
      Assume $\varphi'(0) = 1$ and $\varphi'(n + 1) = \varphi'(n) \cdot (n + 1)$
      for all $n \in \Nat$.
      Then $\varphi = \varphi'$.
    \end{lemma}
    \begin{proof}
      Define $\Phi = \{ n \in \Nat \mid \varphi(n) = \varphi'(n) \}$.

      (1) $0 \in \Phi$.
      Indeed $\varphi(0) = 1 = \varphi'(0)$.

      (2) For all $n \in \Phi$ we have $n + 1 \in \Phi$. \\
      Proof.
        Let $n \in \Phi$.
        Then $\varphi(n) = \varphi'(n)$.
        Hence $\varphi(n + 1)
          = \varphi(n) \cdot (n + 1)
          = \varphi'(n) \cdot (n + 1)
          = \varphi'(n + 1)$.
      Qed.

      Thus $\Phi$ contains every natural number.
      Therefore $\varphi(n) = \varphi'(n)$ for every $n \in \Nat$.
    \end{proof}
  \end{forthel}

  \begin{forthel}
    \begin{definition}\printlabel{ARITHMETIC_11_6912805412274176}
      $\fac$ is the map from $\Nat$ to $\Nat$ such that $\fac(0) = 1$ and
      $\fac(n + 1) = \fac(n) \cdot (n + 1)$ for all $n \in \Nat$.
    \end{definition}

    Let $n!$ stand for $\fac(n)$.
  \end{forthel}

  \begin{forthel}
    \begin{lemma}\printlabel{ARITHMETIC_11_5963501742850048}
      Let $n$ be a natural number.
      Then $n \in \dom(\fac)$.
    \end{lemma}
  \end{forthel}

  \begin{forthel}
    \begin{lemma}\printlabel{ARITHMETIC_11_8076250486669312}
      Let $n$ be a natural number.
      Then $n! \in \Nat$.
    \end{lemma}
  \end{forthel}

  \begin{forthel}
    \begin{lemma}\printlabel{ARITHMETIC_11_1159827327811584}
      $0! = 1$.
    \end{lemma}
  \end{forthel}


  \section{Basic properties}

  \begin{forthel}
    \begin{lemma}\printlabel{ARITHMETIC_11_7831802427211776}
      Let $n$ be a natural number.
      Then \[ (n + 1)! = n! \cdot (n + 1) \]
      and \[ (n + 1)! = (n + 1) \cdot n!. \]
    \end{lemma}
  \end{forthel}

  \begin{forthel}
    \begin{proposition}\printlabel{ARITHMETIC_11_6123519005949952}
      Let $n$ be a natural number.
      Then $n! \neq 0$.
    \end{proposition}
    \begin{proof}
      Define \[ \Phi = \class{n' \in \Nat | n'! \neq 0}. \]

      (1) $\Phi$ contains $0$.
      Indeed $0! = 1 \neq 0$.

      (2) For all $n' \in \Phi$ we have $n' + 1 \in \Phi$. \\
      Proof.
        Let $n' \in \Phi$.
        We have $(n' + 1)! = (n' + 1) \cdot n'!$.
        $n' + 1$ and $n'!$ are nonzero.
        Hence $(n' + 1)! \neq 0$.
      Qed.

      Thus $\Phi$ contains every natural number.
      Therefore $n! \neq 0$.
    \end{proof}
  \end{forthel}  
\end{document}
