\documentclass{article}
\begin{document}
  \section{Multiplication}\label{sec:multiplication}

  \begin{forthel}
    %[prove off][check off]
    [readtex \path{dedekind.ftl.tex}]
    [readtex \path{library/source/arithmetics/03_addition.ftl.tex}]
    %[prove on][check on]
  \end{forthel}

  \begin{forthel}
    \begin{lemma}\label{mul_existence}
      There exists a $\varphi : \Nat \times \Nat \to \Nat$ such
      that for all $n \in \Nat$ we have
      \[\varphi(n, 0) = 0\]
      and
      \[\varphi(n, m + 1) = \varphi(n,m) + n\]
      for any $m \in \Nat$.
    \end{lemma}
    \begin{proof}
      Take $A = [\Nat \to \Nat]$.
      Define $a(n) = 0$ for $n \in \Nat$.
      Then $A$ is a set and $a \in A$.

      [skipfail on] % Wrong proof task %!!
      Define $f(g) = \fun n \in \Nat. g(n) + n$ for $g \in A$.
      [skipfail off]

      Then $f : A \to A$.
      Indeed $f(g)$ is a map from $\Nat$ to $\Nat$ for any $g \in A$.
      Consider a $\psi : \Nat \to A$ such that $\psi$ is recursively defined by
      $a$ and $f$ (by \cref{dedekind_existence}).
      For any objects $n, m$ we have $(n,m) \in \Nat \times \Nat$ iff
      $n, m \in \Nat$.
      Define $\varphi(n,m) = \psi(m)(n)$ for $(n,m) \in \Nat \times \Nat$.

      (1) Then $\varphi$ is a map from $\Nat \times \Nat$ to $\Nat$.
      Indeed $\varphi(n,m) \in \Nat$ for all $n, m \in \Nat$.

      (2) For all $n \in \Nat$ we have $\varphi(n,0) = 0$. \\
      Proof.
        Let $n \in \Nat$.
        Then $\varphi(n,0)
          = \psi(0)(n)
          = a(0)
          = 0$.
      Qed.

      (3) For all $n, m \in \Nat$ we have $\varphi(n, m + 1) =
      \varphi(n,m) + n$. \\
      Proof.
        Let $n, m \in \Nat$.
        Then $\varphi(n, m + 1)
          = \psi(m + 1)(n)
          = f(\psi(m))(n)
          = \psi(m)(n) + n
          = \varphi(n,m) + n$.
      Qed.

      [prover vampire]
      Hence the thesis (by 1, 2, 3).
      [prover eprover]
    \end{proof}
  \end{forthel}

  \begin{forthel}
    \begin{lemma}\label{mul_uniqueness}
      Let $\varphi, \varphi' : \Nat \times \Nat \to \Nat$.
      Assume that for all $n \in \Nat$ we have
      \[\varphi(n, 0) = 0\]
      and
      \[\varphi(n, m + 1) = \varphi(n,m) + n\]
      for any $m \in \Nat$.
      Assume that for all $n \in \Nat$ we have
      \[\varphi'(n, 0) = 0\]
      and
      \[\varphi'(n, m + 1) = \varphi'(n,m) + n\]
      for any $m \in \Nat$.
      Then $\varphi = \varphi'$.
    \end{lemma}
    \begin{proof}
      (a) Define $\Phi = \{ m \in \Nat \mid \varphi(n,m) = \varphi'(n,m)$ for
      all $n \in \Nat \}$.

      (1) $0 \in \Phi$.
      Indeed $\varphi(n,0) = 0 = \varphi'(n,0)$ for all $n \in \Nat$.

      (2) For all $m \in \Phi$ we have $m + 1 \in \Phi$. \\
      Proof.
        Let $m \in \Phi$.
        Then $\varphi(n,m) = \varphi'(n,m)$ for all $n \in \Nat$.
        Hence $\varphi(n, m + 1)
          = \varphi(n,m) + n
          = \varphi'(n,m) + n
          = \varphi'(n, m + 1)$
        for all $n \in \Nat$.
        Thus $m + 1 \in \Phi$ (by a).
      Qed.

      Thus $\Phi$ contains every natural number (by \cref{ARITHMETIC_03_647949900054528}).
      Therefore $\varphi(n,m) = \varphi'(n,m)$ for all $n, m \in \Nat$.
      Consequently $\varphi = \varphi'$.
      Indeed $\varphi(a) = \varphi'(a)$ for all $a \in \Nat \times \Nat$.
    \end{proof}
  \end{forthel}
\end{document}
