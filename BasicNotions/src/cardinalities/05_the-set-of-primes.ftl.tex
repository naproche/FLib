\documentclass[../../basic-notions.ftl.tex]{subfiles}

\begin{document}

  \typeout{
    \begin{forthel}
      [readtex FLib/BasicNotions/sections/tuples/01_tuples.ftl.tex]

      [checkconsistency off]

      Let $x,y,z$ denote sets.
    \end{forthel}
  }


  \section{The set of prime numbers}

  \begin{forthel}
    \begin{lemma}
      The class of prime numbers is a set.
    \end{lemma}
    \begin{proof}
      Let $P$ be the class of prime numbers.
      Then $P = \{ n \in \mathbb{N} \mid n "is prime" \}$.
      Hence $P$ is a set (by separation).
    \end{proof}

    \begin{definition}
      $\mathbb{P}$ is the class that is the set of prime numbers.
    \end{definition}

    Let the set of prime numbers stand for $\mathbb{P}$.

    \begin{theorem}[Euclid]
      The set of prime numbers is countably infinite.
    \end{theorem}
    \begin{proof}
      $\mathbb{P}$ is a subset of $\mathbb{N}$.
      Hence $\mathbb{P}$ is couontable.

      Let us show that $\mathbb{P}$ is infinite.
        Assume the contrary.
        Take a natural number $m$ such that $\mathbb{P}$ has $m$ elements.
        Consider a bijection $p$ between $\set{1, \dots, m}$ and $\mathbb{P}$.
        Then every prime number is equal to $\val{p}{i}$ for some $i \in \set{1, \dots, m}$.

        Define $$P(i) =
          \begin{cases}
            i = 1 & \val{p}{1} \\
            i > 1 & \val{p}{i} \cdot P(i - 1)
          \end{cases}$$
        for $i \in \set{1, \dots, m}$.

        (1) Take $n = P(m) + 1$.

        (2) Consider a prime divisor $q$ of $n$.

        Let us show that (3) $q$ divides $P(m)$.
          Define $N = \{ "natural number" k \mid "if" k \in \set{1, \dots, m} "then" \val{p}{i} "divides" P(k) "for every" i \in \set{1, \dots, k} \}$.

          (BASE CASE) $N$ contains $0$.
          Indeed $0 \notin \set{1, \dots, m}$.

          (INDUCTION STEP) For all natural numbers $k$ we have $k \in N \implies k + 1 \in N$. \\
          Proof.
            Let $k$ be a natural number.
            Assume $k \in N$.
            Suppose $k + 1 \in \set{1, \dots, m}$.

            Case $k + 1 = 1$.
              Then $P(k + 1) = \val{p}{1}$.
              We have $\set{1, \dots, 1} = \set{1}$.
              Hence $\val{p}{i}$ divides $P(k + 1)$ for every $i \in \set{1, \dots, 1}$.
            End.

            Case $k + 1 \neq 1$.
              Then $k \in \set{1, \dots, m}$.
              Hence $\val{p}{i} \mid P(k)$ for every $i \in \set{1, \dots, k}$.
              Let $i \in \set{1, \dots, k + 1}$.
              We have $P(k + 1) = \val{p}{k + 1} \cdot P(k)$.

              Case $i = k + 1$.
                Then $\val{p}{i} = \val{p}{k + 1} \mid \val{p}{k + 1}$.
                Hence $\val{p}{i} \mid \val{p}{k + 1} \cdot P(k) = P(k + 1)$.
              End.

              Case $i < k + 1$.
                Then $i \leq k$.
                Hence $\val{p}{i} \mid P(k)$.
                Thus $\val{p}{i} \mid \val{p}{k + 1} \cdot P(k) = P(k + 1)$.
              End.
            End.
          Qed.

          Hence every natural number is contained in $N$.
          Thus $m \in N$.
          We have $m \in \set{1, \dots, m}$.
          Therefore $q$ divides $P(m)$.
          Indeed $q = \val{p}{i}$ for some $i \in \set{1, \dots, m}$.
        End.

        $q$ is a divisor of $P(m)$ and $q$ is a divisor of $P(m) + 1$ (by 1, 2, 3).
        Hence $q$ is a divisor of $1$ (by NN 03 01 695362).
        Thus $q = 1$.
        Contradiction.
      End.
    \end{proof}
  \end{forthel}
\end{document}
