\documentclass[../../tuples-and-cartesian-products.ftl.tex]{subfiles}

\begin{document}

  \typeout{
    \begin{forthel}
      [readtex FLib/BasicNotions/src/sets-and-functions/sections/02_functions/01_functions.ftl.tex]
      [readtex FLib/BasicNotions/src/natural-numbers/sections/02_ordering/01_ordering.ftl.tex]

      [checkconsistency off]

      [synonym tuple/-s]
    \end{forthel}
  }


  \section{Finite sets}

  \subsection{Initial segments of the natural numbers}

  \begin{forthel}
    \begin{lemma}
      Let $n,m$ be natural numbers.
      Then there exists a set $x$ such that $x = \{ "natural number" k \mid n \leq k \leq m \}$.
    \end{lemma}
    \begin{proof}
      Define $P = \{ "natural number" l \mid "there exists a set" x "such that" x = \{ "natural number" k \mid k \leq l \} \}$.

      (BASE CASE) $P$ contains $0$.
      Proof.
        Take $x = \set{0}$.
        Every natural number that is less than or equal to $0$ is equal to $0$.
        Hence $x = \{ "natural number" k \mid k \leq 0 \}$.
        Thus $0 \in P$.
      Qed.

      (INDUCTION STEP) For all natural numbers $l$ we have $l \in P \implies l + 1 \in P$. \\
      Proof.
        Let $l$ be a natural number.
        Assume $l \in P$.
        Then we can take a set $x'$ such that $x' = \{ "natural number" k \mid k \leq l \}$.
        Take $x = x \cup \set{l + 1}$.
        For all natural numbers $k$ we have $k \leq l + 1$ iff $k \leq l$ or $k = l + 1$.
        Hence $x = \{ "natural number" k \mid k \leq l + 1 \}$.
        Thus $l + 1 \in P$.
        Indeed $x$ is a set.
      Qed.

      Hence every natural number is contained in $P$.
      Thus we can take a set $x'$ such that $x' = \{ "natural number" k \mid k \leq m \}$.
      Define $x = \{ "natural number" k \mid k \in x "and" n \leq k \}$.
      Then every element of $x$ is contained in $x'$.
      Hence $x$ is a set (by separation).
      We have $x = \{ "natural number" k \mid n \leq k \leq m \}$.
    \end{proof}

    \begin{definition}
      Let $n,m$ be natural numbers.
      $\set{n, \dots, m}$ is the set $x$ such that $x = \{ "natural number" k \mid n \leq k \leq m \}$.
    \end{definition}
  \end{forthel}


  \subsection{Finite sets}

  \begin{forthel}
    \begin{definition}
      $x$ has $n$ elements iff $x$ is equipollent to $\set{1, \ldots, n}$.
    \end{definition}

    Let $#x = n$ stand for $x$ has $n$ elements.

    \begin{definition}
      $x$ is finite iff there exists a natural number $n$ such that $x$ has $n$ elements.
    \end{definition}

    Let $#x < \infty$ stand for $x$ is finite.

    \begin{definition}
      A system of finite sets is a system of set $X$ such that every element of $X$ is finite.
    \end{definition}

    \begin{proposition}[6699480602640384]
      Let $x$ be finite and $y$ be a subset of $x$.
      Then $y$ is finite.
    \end{proposition}

    \begin{proposition}[7791524286824448]
      $\emptyet$ is finite.
    \end{proposition}

    \begin{proposition}[8943912055996416]
      Let $u$ be an element.
      Then $\set{u}$ is finite.
    \end{proposition}

    \begin{proposition}[6308309231468544]
      Let $u,v$ be elements.
      Then $\set{u,v}$ is finite.
    \end{proposition}

    \begin{proposition}[4629656516952064]
      Let $x$ be finite.
      Then $x \cap y$ and $y \cap x$ are finite.
    \end{proposition}

    \begin{proposition}[7247472361472000]
      Let $x,y$ be finite.
      Then $x \cup y$ is finite.
    \end{proposition}

    \begin{proposition}[1613072347168768]
      Let $x,y$ be finite.
      Then $x \times y$ is finite.
    \end{proposition}

    \begin{proposition}[2155195855273984]
      Let $x$ be finite.
      Then $\pow(x)$ is finite.
    \end{proposition}

    \begin{proposition}[7884579446718464]
      Let $X$ be a nonempty system of sets.
      Assume that some element of $X$ is finite.
      Then $\bigcap X$ is finite.
    \end{proposition}

    \begin{proposition}[6578041140543488]
      Let $X$ be a finite system of finite sets.
      Then $\bigup X$ is finite.
    \end{proposition}

    \begin{proposition}[8660803444015104]
      Let $f$ be a function from $x$ to $y$ and $a$ be a finite set.
      Then $f[a]$ is finite.
    \end{proposition}

    \begin{corollary}[2307651954278400]
      Let $f$ be a function from $x$ to $y$ and $x$ be finite.
      Then $\range(f)$ is finite.
    \end{corollary}

    \begin{proposition}[5577541272207360]
      Let $f$ be a function from $x$ to $y$ and $b$ be a finite set.
      If $x$ is finite then $f^{-}[b]$ is finite.
    \end{proposition}
  \end{forthel}
\end{document}
