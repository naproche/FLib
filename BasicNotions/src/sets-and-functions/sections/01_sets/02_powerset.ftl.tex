\documentclass[../../sets-and-functions.ftl.tex]{subfiles}

\typeout{
  \begin{forthel}
    [readtex FLib/BasicNotions/src/sets-and-functions/sections/01_sets/01_sets.ftl.tex]

    [checkconsistency off]

    Let $u,v,w$ denote elements.
    Let $x,y,z$ denote sets.
  \end{forthel}
}

\begin{document}
  \section{The powerset}

  \begin{forthel}
    \begin{axiom}
      There exists a set $z$ such that $z = \{ "set" y \mid y \subseteq x \}$.
    \end{axiom}

    \begin{definition}
      $\pow(x)$ is the set $z$ such that $z = \{ "set" y \mid y \subseteq x \}$.
    \end{definition}

    Let the powerset of $x$ stand for $\pow(x)$.

    \begin{proposition}[SF 01 02 481481]
      $\emptyset$ and $x$ are elements of $\pow(x)$.
    \end{proposition}
    \begin{proof}
      We have $\emptyset, x \subseteq x$.
      Hence the thesis.
    \end{proof}

    \begin{corollary}[SF 01 02 671341]
      $\pow(x)$ is nonempty.
    \end{corollary}

    \begin{proposition}[SF 01 02 833606]
      $\pow(x)$ is a system of subsets of $x$.
    \end{proposition}

    \begin{proposition}[SF 01 02 706547]
      $\bigcup \pow(x) = x$.
    \end{proposition}
    \begin{proof}
      Every element of $\pow(x)$ is a subset of $x$.
      Hence $\bigcup \pow(x) \subseteq x$.

      We have $x \in \pow(x)$.
      Hence every element of $x$ is an element of some element of $\pow(x)$.
      Thus every element of $x$ belongs to $\bigcup \pow(x)$.
      Therefore $x \subseteq \bigcup \pow(x)$.

      Then we have the thesis.
    \end{proof}

    \begin{proposition}[SF 01 02 818609]
      $\bigcap \pow(x) = \emptyset$.
    \end{proposition}
    \begin{proof}
      We have $\emptyset \in \pow(x)$.
      Hence every element of $\bigcap \pow(x)$ is an element of $\emptyset$.
      Thus $\bigcap \pow(x)$ is empty.
      Therefore $\bigcap \pow(x) = \emptyset$.
    \end{proof}
  \end{forthel}
\end{document}
