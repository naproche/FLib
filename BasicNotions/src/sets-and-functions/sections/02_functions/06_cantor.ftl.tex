\documentclass[../../sets-and-functions.ftl.tex]{subfiles}

\typeout{
  \begin{forthel}
    [readtex FLib/BasicNotions/src/sets-and-functions/sections/01_sets/02_powerset.ftl.tex]
    [readtex FLib/BasicNotions/src/sets-and-functions/sections/02_functions/01_functions.ftl.tex]

    [checkconsistency off]

    Let $u,v,w$ denote elements.
    Let $x,y,z$ denote sets.
    Let $f,g,h$ denote functions.
  \end{forthel}
}

\begin{document}
  \section{Cantor's theorem}

  \begin{forthel}
    \begin{theorem}[Cantor]
      Let $x$ be a set.
      There exists no function from $x$ onto $\pow(x)$.
    \end{theorem}
    \begin{proof}
      Assume the contrary.
      Take a function $f$ from $x$ onto $\pow(x)$.
      Define $N = \{ u \in x \mid u \notin f(u) \}$.
      Then $N$ is a subset of $x$ (by separation).
      Hence $N \in \pow(x)$.
      Thus we can take an element $u$ of $x$ such that $f(u) = N$.
      Then $u \in N$ iff $u \in f(u)$ iff $u \notin N$.
      Contradiction.
    \end{proof}
  \end{forthel}
\end{document}
