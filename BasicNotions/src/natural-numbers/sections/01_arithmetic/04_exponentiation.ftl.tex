\documentclass[../../natural-numbers.ftl.tex]{subfiles}

\begin{document}

  \typeout{
    \begin{forthel}
      [readtex FLib/BasicNotions/src/natural-numbers/sections/01_arithmetic/03_multiplication.ftl.tex]

      [checkconsistency off]

      Let $k, l, m, n$ denote natural numbers.
    \end{forthel}
  }


  \section{Exponentiation}

  \subsection{Axioms}

  \begin{forthel}
    \begin{signature}
      $n^{m}$ is a natural number.
    \end{signature}

    Let the square of $n$ stand for $n^{2}$.
    Let the cube of $n$ stand for $n^{3}$.

    \begin{axiom}[1st exponentiation axiom]
      $n^{0} = 1$.
    \end{axiom}

    \begin{axiom}[2nd exponentiation axiom]
      $n^{m + 1} = n^{m} \cdot n$.
    \end{axiom}
  \end{forthel}


  \subsection{Computation laws}

  \begin{forthel}
    \begin{proposition}[NN 01 04 876526]
      Assume that $n \neq 0$.
      Then
      $$0^{n} = 0.$$
    \end{proposition}
    \begin{proof}
      Take a natural number $m$ such that $n = m + 1$.
      Then

      $  0^{n}$
      $= 0^{m + 1}$       % definition of m
      $= 0^{m} \cdot 0$   % 2nd exponentiation axiom
      $= 0$.              % 1st multiplication axiom
    \end{proof}


    \begin{proposition}[NN 01 04 577060]
      For all natural numbers $n$ we have
      $$1^{n} = 1.$$
    \end{proposition}
    \begin{proof}
      Define $P = \{ "natural number" n | 1^{n} = 1 \}$.

      (BASE CASE) $P$ contains $0$.

      (INDUCTION STEP) For all natural numbers $n$ we have $n \in P \implies n + 1 \in P$. \\
      Proof.
        Let $n$ be a natural number.
        Assume $n \in P$.
        Then

        $  1^{n + 1}$
        $= 1^{n} \cdot 1$   % 2nd exponentiation axiom
        $= 1 \cdot 1$       % P holds for n
        $= 1$.              % 1 is neutral wrt. multiplication
      Qed.

      Hence every natural number is contained in $P$ .
    \end{proof}


    \begin{proposition}[NN 01 04 848167]
      $n^{1} = n$.
    \end{proposition}
    \begin{proof}
      $  n^{1}$
      $= n^{0 + 1}$       % 1st addition axiom
      $= n^{0} \cdot n$   % 2nd exponentiation axiom
      $= 1 \cdot n$       % 1st exponentiation axiom
      $= n$.              % 1 is neutral wrt. multiplication
    \end{proof}


    \begin{proposition}[NN 01 04 846549]
      $n^{2} = n \cdot n$.
    \end{proposition}
    \begin{proof}
      $  n^{2}$
      $= n^{1 + 1}$       % definition of 2
      $= n^{1} \cdot n$   % 2nd exponentiation axiom
      $= n \cdot n$.      % exponentiation law for 1 as exponent
    \end{proof}


    \begin{proposition}[NN 01 04 461164]
      For all $n,m,k$ we have
      $$k^{n + m} = k^{n} \cdot k^{m}.$$
    \end{proposition}
    \begin{proof}
      Define $P = \{ "natural number" k : n^{m + k} = n^{m} \cdot n^{k} "for all natural"\\"numbers" n,m \}$.

      (BASE CASE) $P$ contains $0$. \\
      Proof.
        Let us show that for all natural numbers $n,m$ we have $n^{m + 0} = n^{m} \cdot n^{0}$.
          Let $n,m$ be natural numbers.
          Then

          $  n^{m + 0}$
          $= n^{m}$               % 1st addition axiom
          $= n^{m} \cdot 1$       % 1 is neutral wrt. multiplication
          $= n^{m} \cdot n^{0}$.  % 1st exponentiation axiom
        End.
      Qed.

      (INDUCTION STEP) For all natural numbers $k$ we have $k \in P \implies k + 1 \in P$. \\
      Proof.
        Let $k$ be a natural number.
        Assume $k \in P$.

        Let us show that for all natural numbers $n,m$ we have $n^{m + (k + 1)} =
        n^{m} \cdot n^{k + 1}$.
          Let $n,m$ be natural numbers.
          Then

          $$  n^{m + (k + 1)}$$
          $$= n^{(m + k) + 1}$$               % 2nd addition axiom
          $$= n^{m + k} \cdot n$$             % 2nd exponentiation axiom
          $$= (n^{m} \cdot n^{k}) \cdot n$$   % k \in P
          $$= n^{m} \cdot (n^{k} \cdot n)$$   % associativity of multiplication
          $$= n^{m} \cdot n^{k + 1}.$$        % 2nd exponentiation axiom
        End.
      Qed.

      Hence every natural number is contained in $P$.
    \end{proof}


    \begin{proposition}[NN 01 04 531499]
      For all $n,m,k$ we have
      $$k^{n \cdot m} = (k^{n})^{m}.$$
    \end{proposition}
    \begin{proof}
      Define $P = \{ "natural number" k : n^{m \cdot k} = (n^{m})^{k} "for all natural"\\"numbers" n,m \}$.

      (BASE CASE) $P$ contains $0$.
      Indeed $(n^{m})^{0} = 1 = n^{0} = n^{m \cdot 0}$ for all natural numbers $n,m$.

      (INDUCTION STEP) For all natural numbers $k$ we have $k \in P \implies k + 1 \in P$. \\
      Proof.
        Let $k$ be a natural number.
        Assume $k \in P$.

        For all natural numbers $n,m$ we have $(n^{m})^{k + 1} =
        n^{m \cdot (k + 1)}$. \\
        Proof.
          Let $n,m$ be natural numbers.
          Then

          $$  (n^{m})^{k + 1}$$
          $$= (n^{m})^{k} \cdot n^{m}$$   % 2nd exponentiation axiom
          $$= n^{m \cdot k} \cdot n^{m}$$ % k \in P
          $$= n^{(m \cdot k) + m}$$       % exponentiation law for sum as exponent
          $$= n^{m \cdot (k + 1)}.$$      % 2nd multiplication axiom
        Qed.
      Qed.

      Therefore every natural number is contained in $P$.
    \end{proof}


    \begin{proposition}[NN 01 04 644237]
      For all natural numbers $n,m,k$ we have
      $$((n \cdot m)^{k}) = n^{k} \cdot m^{k}.$$
    \end{proposition}
    \begin{proof}
      Define $P = \{ "natural number" k : (n \cdot m)^{k} = n^{k} \cdot m^{k} "for all natural"\\"numbers" n,m \}$.

      (BASE CASE) $P$ contains $0$.
      Indeed $((n \cdot m)^{0}) = 1 = 1 \cdot 1 = n^{0} \cdot m^{0}$ for all natural numbers $n,m$.

      (INDUCTION STEP) For all natural numbers $k$ we have $k \in P \implies k + 1 \in P$. \\
      Proof.
        Let $k$ be a natural number.
        Assume $k \in P$.

        $((n \cdot m)^{k + 1}) = n^{k + 1} \cdot m^{k + 1}$ for all natural numbers $n,m$. \\
        Proof.
          Let $n,m$ be natural numbers.

          (Claim) We have

          $$  (n^{k} \cdot m^{k}) \cdot (n \cdot m)$$
          $$= ((n^{k} \cdot m^{k}) \cdot n) \cdot m$$  % associativity of multiplication
          $$= (n^{k} \cdot (m^{k} \cdot n)) \cdot m$$  % associativity of multiplication
          $$= (n^{k} \cdot (n \cdot m^{k})) \cdot m$$  % commutativity of multiplication
          $$= ((n^{k} \cdot n) \cdot m^{k}) \cdot m$$  % associativity of multiplication
          $$= (n^{k} \cdot n) \cdot (m^{k} \cdot m).$$ % associativity of multiplication

          Hence

          $$  (n \cdot m)^{k + 1}$$
          $$= (n \cdot m)^{k} \cdot (n \cdot m)$$       % 2nd exponentiation axiom
          $$= (n^{k} \cdot m^{k}) \cdot (n \cdot m)$$   % induction hypothesis
          $$= (n^{k} \cdot n) \cdot (m^{k} \cdot m)$$   % claim
          $$= n^{k + 1} \cdot m^{k + 1}.$$              % 2nd exponentiation axiom
        Qed.
      Qed.

      Therefore every natural number is contained in $P$.
    \end{proof}


    \begin{proposition}[NN 01 04 857078]
      For all $n,m$ we have
      $$n^{m} = 0 \iff (n = 0 "and" m \neq 0).$$
    \end{proposition}
    \begin{proof}
      (1) For all $n,m$ if $n^{m} = 0$ then $n = 0$ and $m \neq 0$. \\
      Proof.
        Define $P = \{ "natural number" m : "for all natural numbers" n "if" n^{m} = 0 "then" n = 0 "and" m \neq 0 \}$.

        (BASE CASE) $P$ contains $0$.
        Indeed for all natural numbers $n$ if $n^{0} = 0$ then we have a contradiction.

        (INDUCTION STEP) For all natural numbers $m$ we have $m \in P \implies m + 1 \in P$. \\
        Proof.
          Let $m$ be a natural number.
          Assume $m \in P$.

          For all natural numbers $n$ if $n^{m + 1} = 0$ then $n = 0$ and
          $m + 1 \neq 0$. \\
          Proof.
            Let $n$ be a natural number.
            Assume $n^{m + 1} = 0$.
            Then $0 = n^{m + 1} = n^{m} \cdot n$.
            Hence $n^{m} = 0$ or $n = 0$.
            We have $m + 1 \neq 0$ and if $n^{m} = 0$ then $n = 0$.
            Hence the thesis.
          Qed.
        Qed.

        Thus every natural number is contained in $P$.
      Qed.

      (2) For all $n,m$ if $n = 0$ and $m \neq 0$ then $n^{m} = 0$. \\
      Proof.
        Let $n,m$ be natural numbers.
        Assume $n = 0$ and $m \neq 0$.
        Take a natural number $k$ such that $m = k + 1$.
        Then

        $$  n^{m}$$
        $$= n^{k + 1}$$       % definition of m
        $$= n^{k} \cdot n$$   % 2nd exponentiation axiom
        $$= 0^{k} \cdot 0$$   % n = 0
        $$= 0.$$              % 1st multiplication axiom
      Qed.
    \end{proof}
  \end{forthel}
\end{document}
