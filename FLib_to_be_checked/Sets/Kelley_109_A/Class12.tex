\documentclass{scrartcl}
\usepackage[english]{babel}
\usepackage{enumerate, latexsym, amssymb, amsmath, amsthm}
\usepackage{framed, multicol}
\newenvironment{forthel}{\begin{leftbar}}{\end{leftbar}}

%%%%%%%%%% Start TeXmacs macros
\newcommand{\tmaffiliation}[1]{\\ #1}
\newcommand{\tmem}[1]{{\em #1\/}}
\newenvironment{enumeratenumeric}{\begin{enumerate}[1.] }{\end{enumerate}}
%\newenvironment{proof}{\noindent\textbf{Proof\ }}{\hspace*{\fill}$\Box$\medskip}
\newenvironment{quoteenv}{\begin{quote} }{\end{quote}}
\newtheorem*{axiom}{Axiom}
\newtheorem*{lemma}{Lemma}
\newtheorem*{theorem}{Theorem}
\newtheorem*{definition}{Definition}
\newtheorem*{signature}{Signature}
\newtheorem*{proposition}{Proposition}
%%%%%%%%%% End TeXmacs macros

\newcommand{\event}{UITP 2018}
\newcommand{\dom}{Dom}
\newcommand{\fun}{aFunction}
\newcommand{\sym}{sym}
\newcommand{\halfline}{{\vspace{3pt}}}
\newcommand{\tab}{{\hspace{1cm}}}
\newcommand{\ball}[2]{B_{#1}(#2)}
\newcommand{\llbracket}{[}
\newcommand{\rrbracket}{]}
\newcommand{\sing}[1]{\{{#1}\}}
\newcommand{\pair}[2]{\{{#1},{#2}\}}
\newcommand{\op}[2]{({#1},{#2})}
\newcommand{\less}[1]{<_{#1}}
\newcommand{\greater}[1]{>_{#1}}
\newcommand{\leeq}[1]{{\leq}_{#1}}
\newcommand{\supr}[1]{\mathrm{sup}_{#1}}
\newcommand{\RR}{\mathbb{R}}
\newcommand{\QQ}{\mathbb{Q}}
\newcommand{\ZZ}{\mathbb{Z}}
\newcommand{\NN}{\mathbb{N}}

\begin{document}

\title{ELEMENTARY SET THEORY}

\subtitle{An SAD3 Formalisation of the Appendix of \\"General Topology" 
by John L. Kelley \\
Relations and Preliminaries}

\date{November 3, 2018}

\maketitle


\subsection{Preliminaries}

\begin{forthel}

[prove off]

Let $x,y,z$ stand for \emph{classes}.


[object/-s]
\begin{signature}[Ontology] An object is a notion.
Let $a,b,c,d,e$ stand for objects.
\end{signature}

Let $a \in x$ stand for $a$ is an \emph{element} of $x$.

\begin{axiom} Every element of $x$ is an object. \end{axiom}

\begin{axiom}[I]
For each $x$ for each $y$ $x = y$ iff for each $z$ 
$z \in x$ iff $z \in y$.
\end{axiom}

[set/-s]
\begin{definition}[1] A \emph{set} is a class that is an object.
\end{definition}


\begin{definition}[2] $x \cup y = \{\text{object } u \mid u \in x \text{ or } u \in y \}$.
\end{definition}

\begin{definition}[3] $x \cap y = \{\text{object } u \mid u \in x \text{ and } u \in y \}$.
\end{definition}

\begin{definition}[25] 
A \emph{subclass} of $y$ is a class $x$ such that each element of $x$ is an
element of $y$. Let $x \subset y$ stand for $x$ is a subclass of $y$.
Let $x$ is \emph{contained} in $y$ stand for $x \subset y$.
\end{definition}

\begin{theorem}[27] $x = y$ iff $x \subset y$ and $y \subset x$.
\end{theorem}

\begin{theorem}[28] If $x \subset y$ and $y \subset z$ then $x \subset z$.
\end{theorem}

\begin{signature}[48] $\op{a}{b}$ is an object.
\end{signature}

\begin{definition}[48a]
An \emph{ordered pair} is an object $c$ such that there exist
objects $a$ and $b$ such that $c = \op{a}{b}$.
\end{definition}

\begin{axiom}[55] If $\op{a}{b}=\op{c}{d}$ then
$a = c$ and $b = d$.
\end{axiom}

\end{forthel}

\subsection{Relations}

\begin{forthel}

[/prove]

[relation/-s]
\begin{definition}[56] A \emph{relation} is a class $r$ such that
every element of $r$ is an ordered pair.
\end{definition}

Let $r,s,t$ stand for relations.

\begin{definition}[57] 
$r \circ s =
\{\op{x}{z} \mid x,z \text{ are objects and there exists } b \text{ such that }
\op{x}{b} \in s \text{ and } \op{b}{z} \in r\}$.
\end{definition}

\begin{theorem}[58] $(r \circ s) \circ t = r \circ (s \circ t)$.
\end{theorem}
\begin{proof} $(r \circ s) \circ t \subset r \circ (s \circ t)$ and
$r \circ (s \circ t) \subset (r \circ s) \circ t$.
\end{proof}


\begin{theorem}[59a] $r \circ (s \cup t) = (r \circ s) \cup (r \circ t)$.
\end{theorem}
\begin{proof} $r \circ (s \cup t) \subset (r \circ s) \cup (r \circ t)$.
$(r \circ s) \cup (r \circ t) \subset r \circ (s \cup t)$.
\end{proof}

\begin{theorem}[59b] $r \circ (s \cap t) \subset (r \circ s) \cap (r \circ t)$.
\end{theorem}

\begin{definition}[60] $r^{-1} = \{\op{b}{a} \mid a,b \text{ are objects and }
\op{a}{b} \in r\}$.
Let the \emph{relation inverse} to $r$ stand for $r^{-1}$.
\end{definition}

\begin{lemma} $r^{-1}$ is a relation.\end{lemma}

\begin{theorem}[61] $(r^{-1})^{-1} = r$.
\end{theorem}
\begin{proof}
$r \subset (r^{-1})^{-1}$.
$(r^{-1})^{-1} \subset r$.
\end{proof}

\begin{lemma}[62a] Assume $r \subset s$. Then $r^{-1} \subset s^{-1}$.
\end{lemma}

\begin{lemma}[62b] $(r \circ s)^{-1} \subset (s^{-1}) \circ (r^{-1})$.
\end{lemma}
\begin{proof} Let $w \in (r \circ s)^{-1}$.
Then $w \in (s^{-1}) \circ (r^{-1})$.
\end{proof}

\begin{lemma} $(s^{-1}) \circ (r^{-1}) \subset (r \circ s)^{-1}$.
\end{lemma}
\begin{proof}
$((s^{-1}) \circ (r^{-1}))^{-1} \subset 
((r^{-1})^{-1}) \circ ((s^{-1})^{-1})$ (by 62b) .
$((s^{-1}) \circ (r^{-1}))^{-1} \subset 
r \circ s$ (by 61) .
$(((s^{-1}) \circ (r^{-1}))^{-1})^{-1} \subset 
(r \circ s)^{-1}$ (by 62a).
$(s^{-1}) \circ (r^{-1}) \subset (r \circ s)^{-1}$ (by 61).
\end{proof}

\begin{theorem}[62] $(r \circ s)^{-1} = (s^{-1}) \circ (r^{-1})$.
\end{theorem}
\begin{proof} $(r \circ s)^{-1} \subset (s^{-1}) \circ (r^{-1})$.
$(s^{-1}) \circ (r^{-1}) \subset (r \circ s)^{-1}$.
\end{proof}


\end{forthel}


\end{document}