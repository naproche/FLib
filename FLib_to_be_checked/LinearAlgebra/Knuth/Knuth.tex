\documentclass[12pt]{article}
\usepackage{amsthm, amsmath, amssymb}
\newtheorem{signature}{Signature}
\newtheorem{signatures}{Signature}
\newtheorem{axiom}{Axiom}
\newtheorem{axioms}{Axioms}
\newtheorem{definition}{Definition}
\newtheorem{lemma}{Lemma}
\newtheorem{signaturep}{Signature}
\newtheorem{axiomp}{Axiom}
\newtheorem{definitionp}{Definition}
\newtheorem{theoremp}{Theorem}
 
\newcommand{\power}{{\cal P}} 
\newcommand{\preimg}[2]{{#1}^{-1}[#2]} 
\newcommand{\Seq}[2]{\{#1,\dots,#2\}}
\newcommand{\Set}[3]{\{#1_{#2},\dots,#1_{#3}\}}
\newcommand{\Product}[3]{\prod_{i=#2}^{#3}{#1}_i}
\newcommand{\subfunc}[2]{{#1}_{#2}}
\newcommand{\CC}{{\Bbb C}}
\newcommand{\RR}{{\Bbb R}}
\newcommand{\QQ}{{\Bbb Q}}
\newcommand{\ZZ}{{\Bbb Z}} 
\newcommand{\NN}{{\Bbb N}}
\newcommand{\NNplus}{{\Bbb N}^+}

\newcommand{\signatureitem}{\item}
\newcommand{\axiomitem}{\item}
\newcommand{\definitionitem}{\item}
\newcommand{\lemmaitem}{\item}

\title{Knuth's Exercise in Technical Writing in ForTheL (\LaTeX Version)}
\author{Peter Koepke}

\begin{document}
\maketitle

\subsubsection*{Preliminaries on Sets (Types)}

\begin{signature} A \emph{type} is a set.
Let $A,B,C$ stand for sets.
Let $T$ stand for a type.
Let $a : t$ stand for $a$ is an element of $t$.
Let $a \in t$ stand for $a$ is an element of $t$.
\end{signature}

Let $x \neq y$ stand for $x != y$.

\begin{signature} The \emph{empty set} is a set that has 
no elements. Let $\emptyset$ stand for the empty set.
\end{signature}

\begin{definition} A \emph{subset of} $B$ is a set $A$ such
that every element of $A$ is an element of $B$. 
Let $A \subseteq B$ stand for $A$ is a subset of $B$.
Let $A$ \emph{is contained in} $B$ stand for $A$ is a subset of $B$.
Let $B$ \emph{contains} A stand for A is a subset of B.
\end{definition}

\subsubsection*{Preliminaries on Natural Numbers}

[synonym number/numbers]
\begin{signature} A \emph{natural number} is a notion.
\end{signature}

\begin{definition} $\NN$ is the set of natural numbers.
Let i,j,k,l,m denote natural numbers.
\end{definition}

\begin{signature} $0$ is a natural number.
\end{signature}

\begin{signature} $1$ is a natural number.
\end{signature}

\begin{signature} $k + l$ is a natural number.
\end{signature}

\begin{signature} $k * l$ is a natural number.
\end{signature}

\begin{axiom} $k -<- k + 1$.
\end{axiom}

\begin{axiom} If $k \neq 0$ then there is $l$ 
such that $k = l + 1$.
\end{axiom}

\begin{axiom} $k + 0 = k$.
\end{axiom}

\begin{axiom} $0 * l = 0$.
\end{axiom}

\begin{axiom} $(k + 1) * l = (k * l) + l$. 
\end{axiom}

\begin{signature} $k \leq l$ is an atom.
\end{signature}

\begin{axiom} $k \leq k$.
\end{axiom}

\begin{axiom} If $i \leq j \leq k$ then $i \leq k$.
\end{axiom}

\begin{axiom} If not $l \leq m$ then $m +1 \leq l$.
\end{axiom} 

\begin{axiom} If $i \leq j$ and $k \leq l$ then $i + k \leq j + l$.
\end{axiom}

\begin{axiom} If $i \leq j$ and $k \leq l$ then $i * k \leq j * l$.
\end{axiom}

\begin{axiom} (Archimedean Property) Assume that for every 
$k \in \NN$
$i + (k * l) \leq j + (k * m)$. Then $l \leq m$.
\end{axiom}

\subsubsection{Preliminaries on Functions and Sequences}

\begin{axiom}[Functional Extensionality] Let $f,g$ be functions 
such that $Dom(f) = Dom(g)$.
If $f[i] = g[i]$ for all $i \in Dom(f)$ then $f = g$.
\end{axiom}

\begin{definition} A \emph{sequence of length} $l$ is a function 
$a$ such that 
$Dom(a) = \{i \in \NN \mid 1 \leq i \leq l\}$.
\end{definition}

\begin{definition} 
$\NN^{l} = \{a \mid a \text{ is a sequence of length } l 
\text{ such that } a[i] \in \NN 
\text{ for all }i \in Dom(a)\}$.
\end{definition}

We fix a dimension or length:

\begin{signature} $n$ is a natural number.
Let $a,b,c,p$ denote elements of $\NN^{n}$.
\end{signature}

\begin{signature} $a ++ b$ is a sequence of length $n$
such that $$(a ++ b)[i] = a[i] + b[i]$$ 
for all $i \in Dom(a)$.
\end{signature}

\begin{signature} $0^{n}$ is a sequence of length $n$ such that
$0^{n}[i] = 0$ for all $i \in Dom(0^{n})$.
\end{signature}

\begin{lemma} $c ++ 0^{n}$ is a sequence of length $n$
and $c = c ++ (0^{n})$.
\end{lemma}
\begin{proof} For all $i \in Dom(c)$ 
$c[i] = (c ++ 0^{n})[i]$. 
$Dom(c) = Dom(c ++ 0^{n})$.
\end{proof}

\begin{signature} $k ** a$ is a sequence of length $n$ such that
$(k ** a)[i] = k * a[i]$ for all $i \in Dom(a)$.
\end{signature}

\begin{lemma} $(l + 1) ** p = (l ** p) ++ p$.
\end{lemma}
\begin{proof} $(l ** p) ++ p \in \NN^{n}$.
Let us show that for all elements $i of Dom(p)$ we have 
$$((l + 1) ** p)[i] = ((l ** p) ++ p)[i].$$
end.
\end{proof}


\subsubsection*{Preliminaries on Special Subsets of $\NN^{n}$} 

\begin{signature} A \emph{submonoid} is a subset $Q$ of $\NN^{n}$
such that $0^{n}\in Q$ and $a ++ b \in Q$ for all $a,b \in Q$.
\end{signature}

\begin{lemma} Let $Q$ be a submonoid and $p \in Q$. 
Then for all natural numbers $k$ $k ** p \in Q$.
\end{lemma}
\begin{proof} (Proof by induction)
Let $k$ be a natural number.

(1) Case $k = 0$. Then $k ** p = 0^{n} \in Q$. end.

(2) Case $k \neq 0$. Take $l$ such that $k = l + 1$. $l -<- k$ and
$l ** p \in Q$.
Then $k ** p = (l + 1) ** p = (l ** p) ++ p \in Q$. end.
\end{proof}

\begin{definition} $A(n) = \{a \in \NN^{n} \mid \text{for all } i,j\  
(\text{if } 1 \leq i \leq j \leq n
\text{ then } a[i] \leq a[j])\}$.
\end{definition}

\begin{signature} Let $P \subseteq \NN^{n}$. The 
\emph{submonoid generated by} $P$ is
a submonoid $Q$ such that $P$ is contained in $Q$ and
$Q$ is contained in every submonoid that contains $P$.
Let $P^{*}$ stand for the submonoid generated by $P$.
\end{signature}

\begin{definition} Let $A,B \subseteq \NN^{n}$.
$A +++ B = \{a ++ b \mid a \in A \text{ and } b \in B\}$.
\end{definition}

\subsubsection*{The Lemma}

\begin{lemma} Let $P,C \subseteq \NN^{n}$. Let $C \neq \emptyset$ and 
$C +++ (P^{*}) \subseteq A(n)$.
Then we have $C,P \subseteq A(n)$.
\end{lemma}
\begin{proof}
We have $C \subseteq A(n)$. Indeed for all $c \in C$
          $$c = c ++ 0^{n} \in (C +++ (P^{*})) \subseteq A(n).$$

Let us show that $P \subseteq A(n)$. 
Let $p \in P$.
Let $i,j$ be natural numbers such that $1 \leq i \leq j \leq n$.
Take $c \in C$. For all $k \in \NN$ we have
$$c ++ (k ** p) \in C +++ (P^{*}) \subseteq A(n)$$ 
and 
$$c[i] + (k * p[i]) \leq c[j] + (k * p[j]).$$
$p[i] \leq p[j]$ [using the  Archimedean Property]. end.
\end{proof}




\end{document}