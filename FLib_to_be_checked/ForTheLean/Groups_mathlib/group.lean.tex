\documentclass[12pt]{article}
\usepackage{amsthm, amsmath}
\newtheorem{signature}{Signature}
\newtheorem{signatures}{Signature}
\newtheorem{axiom}{Axiom}
\newtheorem{axioms}{Axioms}
\newtheorem{definition}{Definition}
\newtheorem{lemma}{Lemma}
\newtheorem{signaturep}{Signature}
\newtheorem{axiomp}{Axiom}
\newtheorem{definitionp}{Definition}
\newtheorem{theoremp}{Theorem}
 
\newcommand{\power}{{\cal P}} 
\newcommand{\preimg}[2]{{#1}^{-1}[#2]} 
\newcommand{\Seq}[2]{\{#1,\dots,#2\}}
\newcommand{\Set}[3]{\{#1_{#2},\dots,#1_{#3}\}}
\newcommand{\Product}[3]{\prod_{i=#2}^{#3}{#1}_i}
\newcommand{\subfunc}[2]{{#1}_{#2}}
\newcommand{\CC}{{\Bbb C}}
\newcommand{\RR}{{\Bbb R}}
\newcommand{\QQ}{{\Bbb Q}}
\newcommand{\ZZ}{{\Bbb Z}} 
\newcommand{\NN}{{\Bbb N}}
\newcommand{\NNplus}{{\Bbb N}^+}

\newcommand{\signatureitem}{\item}
\newcommand{\axiomitem}{\item}
\newcommand{\definitionitem}{\item}
\newcommand{\lemmaitem}{\item}

\title{The Lean library file {\tt group.lean} reformulated in ForTheL and pretty-printed}
\author{Peter Koepke}

\begin{document}
\maketitle

This file is a ForTheL version of the first half of\\
{\footnotesize \tt https://github.com/leanprover/lean/blob/master/library/init/algebra/group.lean}
The second half is similar, with additive instead of multiplicative notation.

\subsubsection*{Preliminaries} 

\begin{signature} A \emph{type} is a class.
Let $\alpha$ stand for a type.
Let $a : t$ stand for $a$ is an element of $t$.
\end{signature}

\subsubsection*{Semigroups}

\begin{signature} A \emph{type with multiplication} is a type.
Let $\alpha$ be a type with multiplication and $a,b : \alpha$. 
$a *^{\alpha} b$ is an element of $\alpha$.
\end{signature}  

[synonym semigroup/-s]
\begin{definition} A \emph{semigroup} is a type with multiplication $\alpha$
such that for all $a,b,c : \alpha$
$(a *^{\alpha} b) *^{\alpha} c = a *^{\alpha} (b *^{\alpha} c)$. 
\end{definition}

\begin{definition} A \emph{commutative semigroup} is a semigroup $\alpha$
such that for all $a,b : \alpha$
$a *^{\alpha} b = b *^{\alpha} a$.
\end{definition}

\begin{definition} A \emph{semigroup with left cancellation} is a semigroup
$\alpha$ such that for all $a,b,c : \alpha$
$a *^{\alpha} b = a *^{\alpha} c \Rightarrow b = c$.
\end{definition}

\begin{definition} A \emph{semigroup with right cancellation} is a semigroup
$\alpha$ such that for all $a,b,c : \alpha$
$a *^{\alpha} b = c *^{\alpha} b \Rightarrow a = c$.
\end{definition}

\begin{signature} A \emph{type with one} is a type.
Assume $\alpha$ is a type with one. $1^{\alpha}$ is an
element of $\alpha$.
\end{signature}

\begin{definition} A \emph{monoid} is a semigroup $\alpha$ such that $\alpha$ is a type
with one and
$\forall a : \alpha$ $1^{\alpha} *^{\alpha} a = a$ and
$\forall a : \alpha$ $a *^{\alpha} 1^{\alpha} = a$.
\end{definition}

\begin{definition} A \emph{commutative monoid} is a monoid that
is a commutative semigroup.\end{definition}

\begin{signature} A \emph{type with inverses} is a type. \end{signature}
\begin{signature} Assume $\alpha$ is a type with inverses and $a : \alpha$.
$a^{-1,\alpha}$ is an element of $\alpha$.
\end{signature}

\subsubsection*{Groups}

\begin{definition} A \emph{group} is a monoid $\alpha$ such that $\alpha$ is a type 
with inverses and 
for all $a : \alpha$ $a^{-1,\alpha} *^{\alpha} a = 1^{\alpha}$.
\end{definition}

\begin{definition} A \emph{commutative group} is a group that is a commutative
monoid.
\end{definition}

\begin{lemma}[mul left comm] Let $\alpha$ be a commutative semigroup.
Then for all $a,b,c : \alpha$
$a *^{\alpha} (b *^{\alpha} c) = b *^{\alpha} (a *^{\alpha} c)$.
\end{lemma}

\begin{lemma}[mul right comm] Let $\alpha$ be a commutative semigroup.
Then for all $a,b,c : \alpha$
$a *^{\alpha} (b *^{\alpha} c) = a *^{\alpha} (c *^{\alpha} b)$.
\end{lemma}

\begin{lemma}[mul left cancel iff] Let $\alpha$ be a 
semigroup with left cancellation. Then for all $a,b,c : \alpha$
$a *^{\alpha} b = a *^{\alpha} c \leftrightarrow b = c$.
\end{lemma}

\begin{lemma}[mul right cancel iff] Let $\alpha$ be a 
semigroup with right cancellation. Then for all $a,b,c : \alpha$
$b *^{\alpha} a = c *^{\alpha} a \leftrightarrow b = c$.
\end{lemma}

Let $\alpha$ denote a group.

\begin{lemma}[inv mul cancel left] For all $a,b : \alpha$
$a^{-1,\alpha} *^{\alpha} (a *^{\alpha} b) = b$.
\end{lemma}

\begin{lemma}[inv mul cancel right] For all $a,b : \alpha$
$a *^{\alpha} (b^{-1,\alpha} *^{\alpha} b) = a$.
\end{lemma}

\begin{lemma}[inv eq of mul eq one] Let $a, b : \alpha$ and
$a *^{\alpha} b = 1^{\alpha}$. Then $a^{-1,\alpha} = b$.
\end{lemma}

\begin{lemma}[one inv] $(1^{\alpha})^{-1,\alpha} = 1^{\alpha}$.
\end{lemma}

\begin{lemma}[inv inv] Let $a : \alpha$. Then
$(a^{-1,\alpha})^{-1,\alpha} = a$.
\end{lemma}

\begin{lemma}[mul right inv] Let $a : \alpha$. Then 
$a *^{\alpha} a^{-1,\alpha} = 1^{\alpha}$.
\end{lemma}

\begin{lemma}[inv inj] Let $a,b : \alpha$ and $a^{-1,\alpha} = b^{-1,\alpha}$.
Then $a = b$.\end{lemma}

\begin{lemma}[group mul left cancel] Let $a,b,c : \alpha$ and
$a *^{\alpha} b = a *^{\alpha} c$. Then $b = c$.
\end{lemma}

\begin{lemma}[group mul right cancel] Let $a,b,c : \alpha$ and
$a *^{\alpha} b = c *^{\alpha} b$. Then $a = c$.
\end{lemma}
\begin{proof} $a = (a *^{\alpha} b) *^{\alpha} b^{-1,\alpha}
= (c *^{\alpha} b) *^{\alpha} b^{-1,\alpha} = c$.
\end{proof}

\begin{lemma}[mul inv cancel left] Let $a,b : \alpha$. Then
$a *^{\alpha} (a^{-1,\alpha} *^{\alpha} b) = b$.
\end{lemma}

\begin{lemma}[mul inv cancel right]
Let $a,b : \alpha$. Then
$(a *^{\alpha} b) *^{\alpha} b^{-1,\alpha} = a$.
\end{lemma}

\begin{lemma}[mul inv rev] Let $a,b : \alpha$. Then
$(a *^{\alpha} b)^{-1,\alpha} = b^{-1,\alpha} *^{\alpha} a^{-1,\alpha}$.
\end{lemma}
\begin{proof} 
$(a *^{\alpha} b) *^{\alpha} (b^{-1,\alpha} *^{\alpha} a^{-1,\alpha})
= 1^{\alpha}$.
\end{proof}

\begin{lemma}[eq inv of eq inv] Let $a,b : \alpha$ and $a = b^{-1,\alpha}$.
Then $b = a^{-1,\alpha}$.
\end{lemma}

\begin{lemma}[eq inv of mul eq one] Let $a,b : \alpha$ and
$a *^{\alpha} b = 1^{\alpha}$. Then $a = b^{-1,\alpha}$.
\end{lemma}

\begin{lemma}[eq mul inv of mul eq] Let $a,b,c : \alpha$ and
$a *^{\alpha} c = b$. Then $a = b *^{\alpha} c^{-1,\alpha}$.
\end{lemma}

\begin{lemma}[eq inv mul of mul eq] Let $a,b,c : \alpha$ and
$b *^{\alpha} a = c$. Then $a = b^{-1,\alpha} *^{\alpha} c$.
\end{lemma}

\begin{lemma}[inv mul eq of eq mul] Let $a,b,c : \alpha$ and
$b = a *^{\alpha} c$. Then $a^{-1,\alpha} *^{\alpha} b = c$.
\end{lemma}

\begin{lemma}[mul inv eq of eq mul] Let $a,b,c : \alpha$ and
$a = c *^{\alpha} b$. Then $a *^{\alpha} b^{-1,\alpha} = c$.
\end{lemma}

\begin{lemma}[eq mul of mul inv eq] Let $a,b,c : \alpha$ and
$a *^{\alpha} c^{-1,\alpha} = b$. Then $a = b *^{\alpha} c$.
\end{lemma}

\begin{lemma}[eq mul of inv mul eq] Let $a,b,c : \alpha$ and
$b^{-1,\alpha} *^{\alpha} a = c$ . Then $a = b *^{\alpha} c$.
\end{lemma}

\begin{lemma}[mul eq of eq inv mul] Let $a,b,c : \alpha$ and
$b = a^{-1,\alpha} *^{\alpha} c$. Then $a *^{\alpha} b = c$.
\end{lemma}

\begin{lemma}[mul eq of eq mul inv] let $a,b,c : \alpha$ and
$a = c *^{\alpha} b^{-1,\alpha}$. Then $a *^{\alpha} b = c$.
\end{lemma}

\begin{lemma}[mul inv] Let $\alpha$ be a commutative group. 
Let $a,b : \alpha$. Then
$(a *^{\alpha} b)^{-1,\alpha} = a^{-1,\alpha} *^{\alpha} b^{-1,\alpha}$.
\end{lemma}


\end{document}