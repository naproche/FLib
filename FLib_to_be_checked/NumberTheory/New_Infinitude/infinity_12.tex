\documentclass{letter}
\usepackage{amsmath,amssymb,bbm}

%%%%%%%%%% Start TeXmacs macros
\newcommand{\precprec}{\prec\!\!\!\prec}
\newcommand{\section}[1]{\medskip\bigskip

\noindent\textbf{\LARGE #1}}
\newcommand{\subsection}[1]{\medskip\bigskip

\noindent\textbf{\Large #1}}
\newcommand{\tmem}[1]{{\em #1\/}}
\newcommand{\tmop}[1]{\ensuremath{\operatorname{#1}}}
\newcommand{\tmstrong}[1]{\textbf{#1}}
\newcommand{\tmtextbf}[1]{{\bfseries{#1}}}
\newcommand{\tmtextit}[1]{{\itshape{#1}}}
\newcommand{\tmtextup}[1]{{\upshape{#1}}}
\newenvironment{itemizeminus}{\begin{itemize} \renewcommand{\labelitemi}{$-$}\renewcommand{\labelitemii}{$-$}\renewcommand{\labelitemiii}{$-$}\renewcommand{\labelitemiv}{$-$}}{\end{itemize}}
\newenvironment{proof}{\noindent\textbf{Proof\ }}{\hspace*{\fill}$\Box$\medskip}
\newtheorem{lemma}{Lemma}
%%%%%%%%%% End TeXmacs macros

\begin{document}

\title{There are infinitely many primes}\author{}\maketitle
















\section{Suggestions:}

Ad hoc extensions to SAD, to be used as proof of concept for a subsequent
project of extensive structural changes and extensions to the ForTheL
language.

Work with the established top level sections of SAD. Complement the official
form with some abbreviated forms. Try to keep the sentence structure close to
the official form, so that the changes to the SAD code are only superficial.

E.g. ``Work with X'' can be derived from the official ``Signature. X is a
notion.'' \ by replacing ``Signature.'' by ``Work with'' and ``is a notion''
by `` ''. Or a listing of axiom can be carried out by sentences of the form ``
- the\_commutativity\_of\_addition: $m + n = n + m$'' instead of ``Axiom
the\_commutativity\_of\_addition. $m + n = n + m$.'' again by replacing \
``Axiom'' by ``-'' and the first ``.'' by ``:''.

These changes have to be be carried out so that the function of each sentence
is uniquely identifiable by human readers and by the system.

Later, the controlled language ForTheL is to be turned into a controlled
language with grammatical correctness requirements and a rich spectrum of
common mathematical phrases.

Examples:

{\tmstrong{Signature extensions}}: Signature. Let ... . Let ... . X is a
notion.

One sentence version: (We) (also) Work with X. Fix the relation ... . The
constant X is a number ... . We require that an element is a ... . The term $f
\left( x \right)$ is a set. The {\tmem{domain}} is ... (Here, the introduction
is signaled by {\tmem{italics}}.

Short version with premises: Given (that) ... fix the relation/constant/term.

{\tmstrong{Axiom}}: Axiom. Let ... . Let ... . Statement.

One sentence version: (We) presuppose that Statement.

Short version with premises: For ... postulate that .... .

List of axioms: `` - the\_commutativity\_of\_addition: $m + n = n + m$'', see
above.

{\tmstrong{Definitions}}: Definition. Let ... . Let ... . Statement.

One sentence version: (We) define .... .

Short version with premises: Define, for ..., t=... . Define, for ..., ... .

List of definitions: `` .....

{\tmstrong{Claims}}:

``Note that .... .'' anstelle von ``We have ....''

``Obviously/Trivially ...'' (To be used when no proof is given)

...

{\tmstrong{Induction.}} SAD handles inductions with a inbuilt special
relation, which nevertheless has to be invoked by a signature extension:

Signature IH. m -<- n is an atom. Axiom IH. m < n => m -<- n.

These commands could be be invoked instead by something like: ``We can carry
out inductions on the relation $<$.''

In the case of natural numbers, the Axiom IH is a version of the induction
axiom (schema) of first-order Peano arithmetic.

\section{I. Foundations}

We work with {\tmem{sets}}. Let $A, X, Y, Z$ denote sets. We also work with
{\tmem{elements}}. Let $x, y, z$ denote elements. We require that an
{\tmem{element}} of $X$ is an element. Let $x$ {\tmem{belongs to}} $X$ stand
for $x$ is an element of $X$. Let $x$ {\tmem{is in }}$X$ stand for $x$ is an
element of $X$. Let $x \in X$ stand for $x$ is an element of $X$. The constant
$\emptyset$ is a set that has no elements. $X$ is {\tmem{empty}} stands for $X
= \emptyset$. $X$ is {\tmem{nonempty}} stands for $X \neq \emptyset$. Define a
{\tmem{subset}} of $Y$ as a set such that every element of $X$ belongs to $Y$.
$X \subseteq Y$ stands for $X$ is a subset of $Y$.

We can show that (a) $X \subseteq X$ (reflexivity). (b) $X \subseteq Y
\subseteq Z \rightarrow X \subseteq Z$ (transitivity). (c) \ $X \subseteq Y
\subseteq X \rightarrow X = Y$ (symmetry).

We also work with functions. Let $f, g, h$ denote functions. The
{\tmem{domain}} of $f$ is a set. Given $x \in \tmop{dom} f$, the term $f
\left( x \right)$ is an element. Let $f_x$ stand for $f \left( x \right)$.
Define for $X \subseteq \tmop{dom} f$ the term $f \left[ X \right] = \left\{ f
\left( x \right)  \left| \right. x \in X \right\}$. Let $\tmop{ran} f$ stand
for $f \left[ \tmop{dom} f \right]$. Let the {\tmem{range}} of $f$ also stand
for $f \left[ \tmop{dom} f \right]$. A function {\tmem{from}} $X$ denotes a
function $f$ such that $\tmop{dom} f = X$. A function {\tmem{from}} $X$
{\tmem{to}} $Y$ denotes a function $f$ such that $\tmop{dom} f = X$ and
$\tmop{ran} f \subseteq Y$. $f : X$ means that $f$ is a function from $X$. $f
: X \rightarrow Y$ means that $f$ is a function from $X$ to $Y$.

Note that if $x \in \tmop{dom} f$ then $f \left( x \right) \in \tmop{ran} f$.

Define for $X \subseteq \tmop{dom} f$ the term $f \upharpoonright X$ to be a
function $g$ from $X$ such that $g \left( x \right) = f \left( x \right)$ for
every $x \in X$.

We work with {\tmem{numbers}}. Let $i, j, k, l, m, n, q, r$ denote numbers.
Define $\mathbbm{N}$ to be the set of numbers. The constant $0$ is a number.
Let $x$ is {\tmem{nonzero}} stand for $x \neq 0$. The constant $1$ is a
nonzero number. The term $m + n$ is a number. Let the {\tmem{sum}} of $m$ and
$n$ stand for $m + n$. The term $m \cdot n$ is a number. Let the {\tmem{sum}}
of $m$ and $n$ stand for $m \cdot n$. We assume the following axioms.
\begin{itemizeminus}
  \item the associativity of addition: $\left( m + n \right) + l = m + \left(
  n + l \right)$;
  
  \item the neutrality of 0: $m + 0 = 0 + m = m$;
  
  \item the commutativity of addition: $m + n = n + m$;
  
  \item the associativity of multiplication: $\left( m \cdot n \right) \cdot l
  = m \cdot \left( n \cdot l \right)$;
  
  \item the neutrality of 1: $m \cdot 1 = 1 \cdot m = m$;
  
  \item $m \cdot 0 = 0 = 0 \cdot m$;
  
  \item the commutativity of multiplication: $m \cdot n = n \cdot m$;
  
  \item distributivity: $m \cdot \left( n + l \right) = \left( m \cdot n
  \right) + \left( m \cdot l \right)$, and $\left( n + l \right) \cdot m =
  \left( n \cdot m \right) + \left( l \cdot m \right)$;
  
  \item additive cancellation: if $l + m = l + n$ or $m + l = n + l$ then $m =
  n$;
  
  \item multiplicative cancellation: Assume that $l$ is nonzero. Then $l + m =
  l + n$ or $m + l = n + l$ implies $m = n$;
  
  \item if $m + n = 0$ then $m = 0$ and $n = 0$.
\end{itemizeminus}
We can show that if $m \cdot n = 0$ then $m = 0$ or $n = 0$.

Define $m \leq n$ iff there exists $l$ such that $m + l = n$. Define for $m
\leq n$ the expression $n - m$ to be a number such that $m + l = n$. Let $m <
n$ stand for $m \neq n$ and $m \leq n$.

Obviously, $m \leqslant m$; $m \leqslant n \leqslant l \rightarrow m \leq l$;
and $m \leq n \leq m \rightarrow m = n$. Furthermore, $m \leq n$ or $n < m$;
for $l < n$ we have $m + l < m + n$ and $l + m < n + m$; and for $m$ nonzero
and $l < n$ we have $m \cdot l < m \cdot n$ and $l \cdot m < n \cdot m$. We
presuppose that for every number $n$ we have $n = 0$ or $n = 1$ or $1 < n$. We
can prove that $m \neq 0$ implies $n \leqslant n \cdot m$. Indeed observe that
$1 \leqslant m$.

We can carry out inductions on the relation $<$.

Define $\left\{ m, \ldots, n \right\} = \left\{ i \in \mathbbm{N} \left|
\right. m \leq i \leq n \right\}$. For a function $f$ such that $\left\{ m,
\ldots, n \right\} \subseteq \tmop{dom} f$ let $\left\{ f_m, \ldots, f_n
\right\} = \left\{ f \left( i \right)  \left| m \leq i \leq n \right.
\right\}$. We say that $f$ {\tmem{lists}} $X$ {\tmem{in}} $n$ {\tmem{steps}}
iff $\tmop{dom} f = \left\{ 1, \ldots, n \right\}$ and $X = \left\{ f_m,
\ldots, f_n \right\}$. $X$ is called {\tmem{finite}} iff there is a function
$f$ and a number $n$ such that $f$ lists $X$ in $n$ steps. $X$ is called
{\tmem{infinite}} iff $X$ is not finite.



\section{II. Prime Numbers}

\subsection{1. Divisibility}

We define that $m$ {\tmem{divides}} $n$, $m \left| n \right. $, \ iff for some
\ $l$ $n = m \cdot l$. A {\tmem{divisor}} of $n$ is defined as a number that
divides $n$. For $m$ nonzero and $m \left| \right. n$, $\frac{n}{m}$ is
defined as a number $l$ such that $n = m \cdot l$. \

Obviously, $l |m| n \rightarrow l|n$ (transitivity of divisibility); and if
$l|m$ and $l|n$ then $l|m + n$. Indeed if $l$ is nonzero then $m + n = l \cdot
( \frac{m}{l} + \frac{n}{l})$.

\begin{lemma}
  DivMin. Let $l|m$ and $l|m + n$. Then $l|n$.
\end{lemma}

\begin{proof}
  Assume that $l, n$ are nonzero. Take $i$ such that $m = l \cdot i$. Take $j$
  such that $m + n = l \cdot j$.
  
  Let us show that $i \leq j$. Assume the contrary. Then $j < i$. $m + n = l
  \cdot j < l \cdot i = m$. $m \leq m + n$. $m = m + n$. $n = 0$.
  Contradiction.
  
  Take $k = j - i$. We have $(l \cdot i) + (l \cdot k) = (l \cdot i) + n$.
  Hence $n = l \cdot k$.
\end{proof}

We can show: if $m|n \neq 0$, then $m \leq n$.

\begin{lemma}
  DivAsso. Let $l$ be nonzero and divide $m$. Then $n \cdot \frac{m}{l} =
  \frac{n \cdot m}{l}$.
\end{lemma}

\begin{proof}
  $(l \cdot n) \cdot \frac{m}{l} = l \cdot \frac{n \cdot m}{l}$.
\end{proof}

Define $\mathbbm{N}^+ = \left\{ n \in \mathbbm{N} \left| \right. n \neq 0
\right\}$.

For $f : \left\{ m, \ldots, n \right\} \rightarrow \mathbbm{N}^+$ consider
$\prod_{i = m}^n f_i$ which is an element of $\mathbbm{N}^+$. We presuppose
that for $f : \{m, \ldots, n\}- > \mathbbm{N}^+$ and $m \leq j \leq n$, $f_j$
divides $\prod_{i = m}^n f_i$.

\subsection{2. Primes}

Let $m$ is {\tmem{trivial}} stand for $m = 0$ or $m = 1$. Let $m$ is
{\tmem{nontrivial}} stand for $m \neq 0$ and $m \neq 1$. We call a number $q$
{\tmem{prime}} iff $q$ is nontrivial and for every divisor $m$ of $q$ we have
$m = 1$ or $m = q$. We say that $m$ is {\tmem{compound}} for $m$ is not prime.

\begin{lemma}
  PrimDiv. Every nontrivial $k$ has a prime divisor.
\end{lemma}

\tmtextbf{Proof} by induction on $k$. Let $k$ be nontrivial. Case $k$ is
prime. Obvious. Case $k$ is compound. Take a divisor $m$ of $k$ such that $m
\neq 1$ and $m \neq k$. $m \neq 0$. $m$ is nontrivial and $m \precprec k$.
Take a prime divisor $n$ of $m$. $n$ is a prime divisor of $k$. end. qed.

{\noindent}\tmtextbf{Theorem \tmtextup{1}. }\tmtextit{The set of prime numbers
is infinite.}{\hspace*{\fill}}{\medskip}

\begin{proof}
  Let $A$ be a finite set of prime numbers. Take a function $p$ and a number
  $r$ such that $p$ lists $A$ in $r$ steps. $\text{\tmop{ran}} p \subseteq
  \mathbbm{N}^+$. $\prod_{i = 1}^r p_i \neq 0$. Take $n = \prod_{i = 1}^r p_i
  + 1$. $n$ is nontrivial. Take a prime divisor $q$ of $n$.
  
  Let us show that $q$ is not an element of $A$. Assume the contrary. Take $i$
  such that ($1 \leq i \leq r$ and $q = p_i$). $p_i$ divides $\prod_{i = 1}^r
  p_i$ (by MultProd). Then $q$ divides $1$ (by DivMin). Contradiction. qed.
  
  Hence $A$ is not the set of prime numbers.
\end{proof}

\end{document}
