\newtheorem{signature}{Signature} 
\newtheorem{axiom}{Axiom} 
\newcommand{\power}{{\cal P}} 
\newcommand{\preimg}[2]{{#1}^{-1}[#2]} 
\newcommand{\Seq}[2]{\{#1,\dots,#2\}}
\newcommand{\Set}[3]{\{#1_{#2},\dots,#1_{#3}\}}
\newcommand{\Product}[3]{\prod_{i=#2}^{#3}{#1}_i}
\newcommand{\subs}[2]{{#1}_{#2}}
\newcommand{\CC}{{\Bbb C}}
\newcommand{\RR}{{\Bbb R}}
\newcommand{\QQ}{{\Bbb Q}}
\newcommand{\ZZ}{{\Bbb Z}} 
\newcommand{\NN}{{\Bbb N}}

\section{There are infinitely many primes}
%#
%# Version 16 March 2012
%#
\section{I. Foundations}


\subsection{1. Sets}

%%[set/-s] [element/-s] [belong/-s] [subset/-s]

\begin{signature}.
A {\bf set} is a notion.
Let $X,Y,Z$ stand for sets. 
\end{signature}

\begin{signature}.
An {\bf element} is a notion.
Let $x,y,z$ stand for elements.
\end{signature}

\begin{signature}.
An {\bf element of} $X$ is a notion.
Let $x$ {\bf belongs to} $X$ stand for $x$ is an element of $X$.
Let $x \in X$ stand for $x$ is an element of $X$.
\end{signature}

\begin{definition} Subset.
A {\bf subset} of $Y$ is a set $X$ such that
every element of $X$ belongs to $Y$.
Let $X \subseteq Y$ stand for $X$ is a subset of $Y$.
\end{definition}

\begin{lemma}.$X \subseteq X$.\end{lemma}

\begin{lemma}.
$X \subseteq Y \subseteq Z  \rightarrow  X \subseteq Z$.
\end{lemma}

\begin{definition}.
$\emptyset$ is a set that has no elements.
Let $X$ is empty stand for $X = \emptyset$.
Let $X$ is nonempty stand for $X \neq \emptyset$.
\end{definition}

\begin{axiom}.
$X \subseteq Y \subseteq X  \rightarrow  X = Y$.
\end{axiom}

\begin{lemma}.
If $X$ has no elements then $X=\emptyset$.
\end{lemma}

\begin{lemma}.$x,y \in \{x,y\}$. If $z \in \{x,y\}$ then $z=x$ or $z=y$.
\end{lemma}

\begin{lemma}.
$\{x,y\}$ is a set.
\end{lemma} 

\begin{definition}.
Assume that every element of $X$ is a set.
$\bigcup X = \{ u | u \text{ is an element of some element of } X\}$.
\end{definition}

\begin{axiom}.
Assume that every element of $X$ is a set.
$\bigcup X$ is a set.
\end{axiom}

\begin{definition}.
$\power(X) = \{\text{set} Y | Y \subseteq X \}$.
\end{definition}

\begin{axiom}.
$\power(X)$ is a set.
\end{axiom}

\subsection{2. Functions}

%%[function/-s]

\begin{signature}.
A {\bf function} is a notion.
Let $f,g,p$ stand for functions.
\end{signature}

\begin{signature}.
$\text{dom} f$ is a set. Let the {\bf domain} of $f$ stand for 
$\text{dom} f$.
\end{signature}

\begin{signature}.
Let $x \in \text{dom} f$. $f(x)$ is an element. Let
$\subs{f}{x}$ stand for $f(x)$.
\end{signature}

%#\begin{definition}.
%#$\preimg{f}{y} = \{ x \in \text{dom} f | f(x) = y \}$.
%#\end{definition}

\begin{definition}.
Let $X \subseteq \text{dom} f$. 
$f[X] = \{f(x) | x \in X \}$.
Let $\text{ran} f$ stand for $f[\text{dom f}]$.
Let the {\bf range} of $f$ stand for $\text{ran} f$.
\end{definition}
 
\begin{definition}.
Let $X,Y$ be sets.
Let a function {\bf from} $X$ {\bf to} $Y$ stand for
a function $f$ such that $\text{dom} f = X$ and
$\text{ran} f \subseteq Y$.
Let $f : X \rightarrow Y$ stand for $f$ is a function
from $X$ to $Y$.
\end{definition}

\begin{lemma}.
Let $x \in \text{dom} f$. $f(x)$ belongs to $\text{ran} f$.
\end{lemma}

\begin{definition}.
Let $X \subseteq \text{dom} f$.
$f \upharpoonright X$ is a function
$g$ such that $\text{dom} g = X$ and for every
$x \in X$ $g(x) = f(x)$.
\end{definition}

\subsection{3. Numbers}

%%[number/-s]

\begin{signature}.
A {\bf number} is a notion. Let $a,b,c,d$ stand for numbers.
\end{signature}

\begin{signature}.
$0$ is a number. Let $a$ is {\bf nonzero} stand for $a \neq 0$.
\end{signature}

\begin{signature}.
$1$ is a nonzero number.
\end{signature}

\subsubsection{3.1. Complex Numbers}

\begin{signature}.
$\CC$ is a set. Let $a$ is {\bf complex} stand for $a \in \CC$.
\end{signature}

\begin{axiom}.
$\CC$ is the set of numbers.
\end{axiom}

\begin{signature}.
$a + b$ is a number. Let the {\bf sum} of $a$ and $b$ stand for $a + b$.
\end{signature}

\begin{signature}.
$a \cdot b$ is a number. Let the {\bf product} of $a$ and $b$ stand for $a \cdot b$.
\end{signature}

\begin{signature}.
$-a$ is a number. Let the {\bf negative} of $a$ stand for $-a$.
\end{signature}

\begin{signature}.
Let $a$ be nonzero. $a^{-1}$ is a number. Let the {\bf inverse}
of $a$ stand for $a^{-1}$.
\end{signature}

\begin{axiom}.
$(a+b)+c = a+(b+c)$.
\end{axiom}
\begin{axiom}.
$a+0 = 0+a = a$.
\end{axiom}
\begin{axiom}.
$a+(-a) = (-a)+a=0$.
\end{axiom}
\begin{axiom}.
$a+b = b+a$.
\end{axiom}
\begin{axiom}.
$(a \cdot b) \cdot c = a \cdot (b \cdot c)$.
\end{axiom}
\begin{axiom}.
$a \cdot 1 = 1 \cdot a = a$.
\end{axiom}
\begin{axiom}.
If $a$ is nonzero then $a \cdot a^{-1} = a^{-1} \cdot a = 1$.
\end{axiom}
\begin{axiom}.
$a \cdot b = b \cdot a$.
\end{axiom}
\begin{axiom}.
$(a + b) \cdot c = (a \cdot c) + (b \cdot c)$.
\end{axiom}

\subsubsection{3.2. Real Numbers}

\begin{signature}.
$\RR$ is a set. Let $a$ is {\bf real} stand for $a \in \RR$.
\end{signature}

\begin{axiom}.
$\RR \subseteq \CC$.
\end{axiom}

%# \begin{lemma}. $\{a \in \CC | a \text{is real}\}$ is a set. \end{lemma}

%# \begin{theorem}.$\RR \subseteq \{a \in \CC | a \text{is real}\}$.\end{theorem}
%#\begin{proof}
%#Let $b$ be an element of $\RR$. 
%#Then $b \in \CC$ and $b$ is real.
%#Hence $b$ belongs to $\{a \in \CC | a \text{is real}\}$.
%#Hence every element of $\RR$ belongs to $\{a \in \CC | a \text{is real}\}$.
%#Hence thesis (by Subset).
%#\end{proof}

\begin{axiom}.
$0,1 \in \RR$.
\end{axiom}

\begin{axiom}.
If $a,b$ are real numbers then $a+b$ is a real number.
\end{axiom}

\begin{axiom}.
If $a,b$ are real numbers then $a \cdot b$ is a real number.
\end{axiom}

\subsubsection{3.3. Rational Numbers}

\begin{signature}.
$\QQ$ is a set. Let $a$ is {\bf rational} stand for $a \in \QQ$.
\end{signature}

\begin{axiom}.
$\QQ \subseteq \RR$.
\end{axiom}

\subsubsection{3.4. Integers}

%%[integer/-s]

\begin{signature}.
$\ZZ$ is a set. Let $a$ is {\bf integer} stand for $a \in \ZZ$.
\end{signature}

Let $r,s$ stand for integer numbers.

\begin{axiom}.
$\ZZ \subseteq \QQ$.
\end{axiom}

\begin{axiom}.
$0,1 \in \ZZ$.
\end{axiom}

\begin{axiom}.
If $r,s$ are integers then $r+s$ is an integer number.
\end{axiom}

\begin{axiom}.
If $r,s$ are integers then $r \cdot s$ is an integer number.
\end{axiom}



\subsubsection{3.5. Natural Numbers}

\begin{signature}.
$\NN$ is a set. Let $a$ is {\bf natural} stand for $a \in \NN$.
Let $i,j,k,m,n,p,q$ stand for natural numbers. 
\end{signature}

\begin{axiom}.
$0 \in \NN$ and for all ($n \in \NN$) $(n+1) \in \NN$.
\end{axiom}

\begin{axiom}.
$0,1 \in \NN$.
\end{axiom}

\begin{axiom}.
If $a,b$ are natural numbers then $a+b$ is a natural number.
\end{axiom}

\begin{axiom}.
If $a,b$ are natural numbers then $a \cdot b$ is a natural number.
\end{axiom}

\subsection{4. Finite Sets and Sequences}

\begin{definition}.
$\Seq{m}{n} =\{ number i  | (m \leq i \leq n) \}$.
\end{definition}

\begin{definition}.
Let $d$ be a function such that $\Seq{m}{n} \subseteq \text{dom} d$. 
$\Set{d}{m}{n} = \{ 1 | i \in \NN \wedge m \leq i \leq n \}$.
\end{definition}

\begin{definition}.
$f$ {\bf enumerates} a set $X$ in $n$ steps iff
$\Seq{1}{n} = \text{dom} f$ and $X=\Set{f}{1}{n}$.
\end{definition}

\begin{definition}.
A set $X$ is {\bf finite} iff there is a function $f$ and a natural number $N$ such that $f$ enumerates $X$ in $n$ steps.
\end{definition}

\begin{definition}.
A set $A$ is {\bf infinite} iff $A$ is not finite.
\end{definition}

\section{II. Prime Numbers}

\subsection{1. Basics}

%%[prime/-s] [compound/-s] [primenumber/-s]

%%[divide/-s] [divisor/-s]

\begin{definition}.
$m$ {\bf divides} $n$ iff for some $l$ $n = m \cdot l$.
Let $m | n$ denote $m$ divides $n$.
Let a {\bf divisor} of $n$ denote a number that divides $n$.
\end{definition}

\begin{definition}.
Assume that $m$ is nonzero and $m | n$.
$\frac{n}{m}$ is a number $l$ such that $n = m \cdot l$.
\end{definition}

\begin{lemma}.
$l | m | n \rightarrow l | n$.\end{lemma}

\begin{lemma}.
Let $l | m$ and $l | n$. Then $l | m + n$.
Indeed if $l$ is nonzero then 
$m + n = l \cdot (\frac{m}{l} + \frac{n}{l})$.
\end{lemma}

\begin{lemma}.
Let $l | m$ and $l | m + n$. Then $l | n$.\end{lemma}
\begin{proof}
Assume that $l,n$ are nonzero.
Take an integer number $p$ such that $m = l \cdot p$. 
Take an integer number $q$ such that $m + n = l \cdot q$.

Let us show that $p \leq q$.
Assume the contrary. Then $q \leq p$.
$m+n = l \cdot q \leq l \cdot p = m$. $m \leq m+n$.
$m = m+n$. $n=0$.
Contradiction. qed. 

Take $r = q - p$.
We have $(l \cdot p) + (l \cdot r) = (l \cdot p) + n$.
Hence $n = l \cdot r$.
\end{proof}


\begin{definition}.
$n$ is {\bf prime} iff $n$ is nontrivial and for every divisor
$m$ of $n$ $m=1 \vee m=n$. Let $n$ is {\bf compound} stand for $n$
is not prime.
\end{definition}

\begin{theorem}.
Every nontrivial natural number has a prime divisor. \end{theorem}
\begin{proof} By induction.
Let $k$ be nontrivial.
Case $k$ is prime. Obvious.
Case $k$ is compound. Obvious.
\end{proof}

\begin{axiom}.
Let $m \leq k \leq n$. Let $d$ be a function such that $\Seq{m}{n} \subseteq \text{dom} d$ and $\text{ran} d \subseteq \NN$.
$d_k$ divides $\Product{d}{m}{n}$.
\end{axiom}

\subsection{2. The Infinitude of Primes}

\begin{theorem}.
The set of prime numbers is infinite.\end{theorem}
\begin{proof}
Let $A$ be a finite set of prime numbers.
Take a function $q$ and a natural number $r$ such that $q$ enumerates
$A$ in $r$ steps.
Take $n=\Product{q}{1}{r}+1$.
Take a prime divisor $p$ of $n$.

Let us show that $p$ is not an element of $A$.
Assume the contrary.
Take $i$ such that ($1 \leq i \leq r$ and $p=\subs{q}{i}$).
$\subs{q}{i}$ divides $\Product{q}{1}{r}$ (by MultProd).
Indeed $\Seq{1}{r} \subseteq \text{dom} q$ and 
$\text{ran} q \subseteq \NN$.
Then $p$ divides $1$ (by DivMin). 
Contradiction. qed.

Hence $A$ is not the set of prime numbers.
\end{proof}


