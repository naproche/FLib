\documentclass[a4paper,10pt]{article}
\usepackage[utf8x]{inputenc}
\usepackage{amsmath}
\usepackage{amssymb}
\usepackage{amsthm}

\theoremstyle{plain}
\newtheorem{theorem}{Theorem}
\newtheorem{lemma}[theorem]{Lemma}

\theoremstyle{definition}
\newtheorem{definition}[theorem]{Definition}
\newtheorem{axiom}[theorem]{Axiom}

%opening
\title{Naproche version of ``There are infinitely many primes''}
\author{Marcos Cramer}

\begin{document}

\maketitle

Assume that there is a set of objects called numbers.
Small latin letters always denote numbers.
Assume that there are numbers $0$ and $1$ and there are functions $+$ and $\cdot$.

\begin{axiom}
$0$ is a number.
\end{axiom}

\begin{definition}
Define $n$ to be nonzero iff $n \neq 0$.
\end{definition}

\begin{axiom}
$1$ is a number.
\end{axiom}

\begin{axiom} 
For all $m$, $n$, $l$, $(m + n) + l = m + (n + l)$.
\end{axiom}

\begin{axiom} 
For all $m$, $m + 0 = m = 0 + m$.
\end{axiom}

\begin{axiom} 
For all $m$, $n$, $m + n = n + m$. 
\end{axiom}

\begin{axiom} 
For all $m$, $n$, $l$, $(m \cdot n) \cdot l = m \cdot (n \cdot l)$.
\end{axiom}

\begin{axiom} 
For all $m$, $m \cdot 1 = m = 1 \cdot m$.
\end{axiom}

\begin{axiom} 
For all $m$, $m \cdot 0 = 0 = 0 \cdot m$.
\end{axiom}

\begin{axiom} 
For all $m$, $n$,$m \cdot n = n \cdot m$.
\end{axiom}

\begin{axiom}
For all $m$, $n$, $l$, $m \cdot (n + l) = (m \cdot n) + (m \cdot l)$ and $(n + l) \cdot m = (n \cdot m) + (l \cdot m)$.
\end{axiom}

\begin{axiom} 
If $l + m = l + n$ or $m + l = n + l$ then $m = n$.
\end{axiom}

\begin{axiom} 
Assume that $l$ is nonzero. If $l \cdot m = l \cdot n$ or $m \cdot l = n \cdot l$ then $m = n$.
\end{axiom}

\begin{axiom} 
If $m + n = 0$ then $m = 0$ and $n = 0$.
\end{axiom}

\begin{lemma} 
If $m \cdot n = 0$ then $m = 0$ or $n = 0$.
\end{lemma}
\begin{proof}
Trivial.
\end{proof}

\begin{definition} 
Define $m \leq n$ iff there exists an $l$ such that $m + l = n$.
\end{definition}

\begin{lemma}
If $m \leq n$, then there is a number $n - m$ such that $m + (n-m) = n$.
\end{lemma}
\begin{proof}
Trivial.
\end{proof}

\begin{lemma} 
For all $m$, $m \leq m$.
\end{lemma}
\begin{proof}
Trivial.
\end{proof}

\begin{lemma} 
$m \leq n \leq l$ implies $m \leq l$.
\end{lemma}
\begin{proof}
Trivial.
\end{proof}

\begin{lemma} 
$m \leq n \leq m$ implies $m = n$. 
\end{lemma}
\begin{proof}
Trivial.
\end{proof}

\begin{definition}
Define $m < n$ iff $m \neq n$ and $m \leq n$.
\end{definition}

\begin{axiom}
For all $m$, $n$, $m \leq n$ or $n < m$. 
\end{axiom}

\begin{lemma} 
Assume that $l < n$.
Then for all $m$, $m + l < m + n$ and $l + m < n + m$.
\end{lemma}
\begin{proof}
Trivial.
\end{proof}

\begin{lemma} 
Assume that $m$ is nonzero and $l < n$.
Then $m \cdot l < m \cdot n$ and $l \cdot m < n \cdot m$.
\end{lemma}
\begin{proof}
Trivial.
\end{proof}

\begin{axiom} 
For all $n$, $n = 0$ or $n = 1$ or $1 < n$.
\end{axiom}

\begin{lemma}
$m \neq 0$ implies $n \leq n \cdot m$.
\end{lemma}
\begin{proof}
Let $m \neq 0$. Then $1 \leq m$.
\end{proof}

\begin{axiom}
Suppose $X$ is a set of numbers such that if for all $m$ such that $m < n$ $m \in X$, then $n \in X$. Then $X$ contains all numbers.
\end{axiom}

\begin{definition}
Define a set $X$ to be finite iff there is a function $F$ and a number $n$ such that for every $x$ in $X$ there is an $m$ such that $m \leq n$ and $F(m) = x$. 
\end{definition}

\begin{definition}
Define a set $X$ to be infinite iff $X$ is not finite.
\end{definition}

\begin{definition} 
Define $m$ to divide $n$ iff there is an $l$ such that $n = m \cdot l$.
\end{definition}

\begin{definition}
Define $m \mid n$ iff $m$ divides $n$.
\end{definition}

\begin{lemma}
Assume that $m$ is nonzero and $m \mid n$. Then there is a number $\frac{n}{m}$ such that $n = m \cdot \frac{n}{m}$.
\end{lemma}
\begin{proof}
Trivial.
\end{proof}

\begin{lemma} 
$l \mid m \mid n$ implies $l \mid n$.
\end{lemma}
\begin{proof}
Trivial.
\end{proof}

\begin{lemma} 
Let $l \mid m$ and $l \mid n$. Then $l \mid m + n$.
\end{lemma}
\begin{proof}
If $l$ is nonzero then $m + n = l \cdot (\frac{m}{l} + \frac{n}{l})$.
\end{proof}

Lemma DivMin. Let $l \mid m$ and $l \mid m + n$. Then $l \mid n$.
\begin{proof}
Assume that $l$ and $n$ are nonzero.
There is an $i$ such that $m = l \cdot i$. 
There is a $j$ such that $m + n = l \cdot j$.

Assume for a contradiction that $j < i$.
$m+n = l \cdot j < l \cdot i = m$. 
$m \leq m+n$.
$m = m+n$. Then $n=0$.
Contradiction. Thus $i \leq j$.
 
Define $k$ to be $j - i$.
We have $(l \cdot i) + (l \cdot k) = (l \cdot i) + n$.
Hence $n = l \cdot k$.
\end{proof}

\begin{lemma} 
Let $m \mid n \neq 0$. Then $m \leq n$.
\end{lemma}
\begin{proof}
Trivial.
\end{proof}

\begin{lemma} 
Suppose $l$ is nonzero and $l \mid m$.
Then $n \cdot \frac{m}{l} = \frac{n \cdot m}{l}$.
\end{lemma}
\begin{proof}
$(l \cdot n) \cdot \frac{m}{l} = l \cdot \frac{n \cdot m}{l}$.
\end{proof}

\begin{axiom}
Suppose that $F$ is a function and $m$ and $n$ are numbers. Then there is a number $\Pi_m^n F$ such that if $m \leq j \leq n$, then $F(j)$ divides $\Pi_m^n F$. If for all $j$ such that $m \leq j \leq n$ $F(j) \neq 0$, then $\Pi_m^n F \neq 0$.
\end{axiom}

Define $m$ to be trivial iff $m = 0$ or $m = 1$.
Define $m$ to be nontrivial iff $m \neq 0$ and $m \neq 1$.

\begin{definition} 
Define $q$ to be prime iff $q$ is nontrivial and
for every $m$ such that $m \mid q$, $m = 1$ or $m = q$.
\end{definition}

Define $m$ to be compound iff $m$ is not prime.

\begin{lemma} 
For every nontrivial $k$ some prime number divides $k$.
\end{lemma}
\begin{proof}
Let $X$ be the set of $k$ such that if $k$ is nontrivial, then some prime number divides $k$.

Let $k$ be a nontrivial number such that for all $m$ such that $m < k$, $m \in X$. There are two cases:

Case 1: $k$ is prime. 
Then $k \in X$.

Case 2: $k$ is compound. 
Then there is an $m$ such that $m \mid k$ and $m \neq 1$ and $m \neq k$.
$m < k$, i.e. $m \in X$. $m$ is nontrivial, so there is a prime number $n$ such that $n$ divides $m$.
Then $n$ divides $k$, so $k \in X$.

So in both cases $k \in X$. Thus $X$ contains all numbers.
\end{proof}

\begin{theorem}
The set of prime numbers is infinite.
\end{theorem}
\begin{proof}
Let $A$ be a finite set of prime numbers.
Then there is a function $P$ and a number $r$ such that for every $n$ in $A$, there is a $k$ such that $k \leq r$ and $P(k) = n$. 
$\Pi_1^n P \neq 0$.
Define $n$ to be $\Pi_1^n P +1$.
$n$ is nontrivial. So there is a prime number $q$ such that $q$ divides $n$.

Assume that $q$ is in $A$. Then there is an $i$ such that $1 \leq i \leq r$ and $q=P(i)$. $P(i)$ divides $\Pi_1^n P$. Then $q$ divides $1$ by lemma DivMin. Contradiction. 

Thus $A$ is not the set of prime numbers.
\end{proof}



\end{document}

